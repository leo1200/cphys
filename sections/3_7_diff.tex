\section{Diffusion}
\thispagestyle{plain}

On a micro scale diffusion is a random walk with the mean free path as the step size.
\textcolor{blue1}{Intuitive example}: Imagine a region with high density. It is more likely that a particle
within this region takes a step out of it than a particle outside stepping in (as of the 
difference in density). Physically, the entropy cannot decrease.

\subsection{Thermodynamic Basics of Diffusion}
\bluebox{How quickly do thermalized, randomly moving particles spread?}
\subsubsection{Mean squared velocity and mean squared relative velocity}
Assume the particles follow a Maxwell-Boltzmann distribution (we assume only one particle species with mass $m$ and temperature $T$)
\begin{equation}
    \begin{gathered}
    f(\vec{v}) d^3 v=\left(\frac{m}{2 \pi k_B T}\right)^{\frac{3}{2}} \exp \left(\frac{-\frac{1}{2} m v^2}{k_B T}\right) d^3 v \\
    f(v) d v=4 \pi v^2\left(\frac{m}{2 \pi k_B T}\right)^{\frac{3}{2}} \exp \left(\frac{-\frac{1}{2} m v^2}{k_B T}\right) d v, \quad v=|\vec{v}|
    \end{gathered}
\end{equation}
Based on the equipartition theorem, the mean squared velocity is
\begin{equation}
    \langle \frac{1}{2} m \vec{v}^2 \rangle = \frac{3}{2} k_B T \quad \rightarrow \quad \langle \vec{v}^2 \rangle = \frac{3k_B T}{m},  \quad v:= \sqrt{\langle \vec{v}^2 \rangle}
\end{equation}
so the mean squared relative velocity is higher (avg here over velocity space)
\begin{equation}
    \langle \vec{v}_{rel}^2 \rangle = \langle (\vec{v} - \vec{v}')^2 \rangle = \langle \vec{v}^2 \rangle + \langle \vec{v'}^2 \rangle - \underbrace{2 \langle \vec{v} \cdot \vec{v'} \rangle}_{=0 \text{ avg. over uncorrelated}} = 2 \langle \vec{v}^2 \rangle = \frac{6 k_B T}{m}
\end{equation}
% = \int (\vec{v}-\vec{v'})^2 f(\vec{v}) f(\vec{v'}) d^3v d^3v'
where we shorthand use $v_{rel} = \sqrt{\langle \vec{v}_{rel}^2 \rangle}$.

\subsubsection{Mean free path and relaxation time}
Consider a fluid with particle density $n$ (in $\m^{-3}$) of particles with effective radius $r$ so effective
cross-section $\sigma = \pi r^2$. In sec. \ref{ssssec:mean_free_path}, we already devised (under particle
movement with the root-mean-square velocity)
\begin{equation}
    \begin{gathered}
        \text{particle moving through stationary particles: } \lambda_{mfp} = \frac{1}{n\sigma} \\
        \text{relative particle movement: } \lambda_{mfp} = \frac{1}{\sqrt{2} n\sigma}
    \end{gathered}
\end{equation}
and the relaxation time (mean time until a particle collides with another)
\begin{equation}
    t_{rel} = \frac{\lambda_{mfp}}{v} = \frac{1}{n\pi r^2} \sqrt{\frac{m}{6k_B T}}
\end{equation}
\redbox{\cite{springel23} - I would say wrongly - uses $\lambda_{mfp} = \frac{1}{n\sigma}$ and with this
defines $t_{scat} = \frac{\lambda_{mfp}}{v}$ and $t_{rel} = \frac{\lambda_{mfp}}{v_{rel}}$, which
is equivalent to our $t_{rel}$ with our $\lambda_{mfp}$.}

\subsubsection{Random Walk | spreading Gaussian distribution in space}
\greenbox{\textbf{Central Limit Theorem:} Let $w(s) ds$ be the probability
of taking a step with length between $s$ and $s+ds$ with mean
$\langle s \rangle$ and variance $\sigma_s^2$. We make $N$ steps,
so the final position along one dimension is
\begin{equation}
    x = s_1 + s_2 + s_3 + \dots + s_N
\end{equation}
By the rules of the mean and variance, one finds
\begin{equation}
    \begin{gathered}
        \langle x \rangle = N \langle s \rangle, \quad \langle s \rangle = \int_{-\infty}^{\infty} s \cdot w(s) \, ds \\
        (\Delta x)^2 = \sigma_x^2 = \langle (x - \langle x \rangle)^2 \rangle = N \sigma_s^2
    \end{gathered}
\end{equation}
And the \textbf{Central Limit Theorem} states
\begin{equation}
    \label{eq:central_limit_theorem}
    \boxed{P(x) = \frac{1}{\sqrt{2\pi}\Delta x} \exp{\left( -\frac{(x-\langle x \rangle)^2}{2(\Delta x)^2} \right)}}
\end{equation}
if the moments $\langle s^n \rangle = \int_{-\infty}^{\infty} s^n w(s) \, ds$ are finite. The basis reasoning
is that one can show that for added random variables their sum converges to one distribution, and as added random
variables are distributed according to the convolution of their distributions and the convolution of two Gaussians is
a Gaussian, the sum of many random variables is a Gaussian.}

Consider particles starting out at the origin with equal probability of
going left or right with $\lambda_{mfp}$. We have
\begin{equation}
    w(s) = \frac{1}{2} \delta(s-\lambda_{mfp}) + \frac{1}{2} \delta(s+\lambda_{mfp}), \quad \langle s \rangle = 0, \quad \sigma_s^2 = \lambda_{mfp}^2
\end{equation}
so therefore, estimating $N = \frac{t}{t_{rel}}$
\begin{equation}
    \langle x \rangle = 0, \quad \langle x^2 \rangle = N \sigma_s^2 = \frac{\lambda_{mfp}^2}{t_{rel}} t \rightarrow \boxed{\sqrt{\langle x^2 \rangle} \propto \sqrt{t}}
\end{equation}
So the distribution flattens and spreads in time. As $\langle (\Delta x)^2 \rangle$ is time dependent, so is $p(x,t)$.

\subsection{Diffusion equation}
\subsubsection{Derivation of the diffusion equation | Fick's law from the microscopic consideration}
Consider the evolution of a system from $t \rightarrow t + \Delta t$. For a change $\Delta t$ let
$\Delta x$ be the according change of a particle position, distributed according to $p(\Delta x, \Delta t)$.
We can therefore write
\begin{equation}
    \begin{gathered}
        n(x,t+\Delta t) = \langle n(x-\Delta x,t) \rangle_{\Delta x \sim p(\cdot,\Delta t)} \underset{\text{Taylor}}{=} n(x,t) - \partial_x n \cdot \langle \Delta x \rangle + \frac{1}{2} \partial_x^2 n \cdot \langle \Delta x^2 \rangle + \dots \\
        \underset{\text{vanishing uneven moments}}{=} n(x,t) + \frac{1}{2} \partial_x^2 n \cdot \langle \Delta x^2 \rangle + \mathcal{O}(\Delta x^4)
    \end{gathered}
\end{equation}
Taylor expansion with respect to time yields
\begin{equation}
    n(x,t+\Delta t) = n(x,t) + \partial_t n \cdot \Delta t
\end{equation}
So combined, we get (ignoring the higher order terms, so it's really an approximation)
\begin{equation}
    \boxed{\partial_t n(x,t) = \mathcolor{blue1}{\frac{\Delta x^2}{2 \Delta t}} \partial_x^2 n(x,t) = \mathcolor{blue1}{D} \partial_X^2 n(x,t)}
\end{equation}
with
\begin{equation}
    \mathcolor{blue1}{\text{diffusion coefficient } D = \frac{\Delta x^2}{2 \Delta t} \propto \frac{\lambda_{mfp}^2}{t_{rel}}}
\end{equation}

We can rewrite this as a conservation equation (introducing the flux $J$) (Fick's law of diffusion)

\begin{equation}
    \partial_t n(x,t) = -\partial_x J(x,t), \quad J(x,t) = -D \partial_x n(x,t)
\end{equation}

where the same principle applies to heat, energy and momentum diffusion.

\subsubsection{Analytical Solution to the diffusion equation via Fourier transform}
Using the Fourier transform in space in the convention
\begin{equation}
    n(x, t)=\frac{1}{2 \pi} \int_{-\infty}^{\infty} \hat{n}(k, t) \exp (-i k x) d k, \quad \hat{n}(k, t)=\int_{-\infty}^{\infty} n(x, t) \exp (i k x) d x
\end{equation}
we yield an ODE in Fourier space with trivial solution
\begin{equation}
    \partial_t \hat{n}(k, t)=-D k^2 \hat{n}(k, t) \quad \rightarrow \quad \hat{n}(k, t)=\hat{n}\left(k, t_0\right) \exp \left(-D k^2\left(t-t_0\right)\right)
\end{equation}
\subsubsubsection{Solution for an initial delta peak in the density over space}
Consider
\begin{equation}
    n\left(x, t_0\right)=N_0 \delta\left(x-x_0\right), \quad \hat{n}\left(x, t_0\right)=N_0 \int_{-\infty}^{\infty} \delta\left(x-x_0\right) \exp (i k x) d x=N_0 \exp \left(i k x_0\right)
\end{equation}
yielding the solution in Fourier space
\begin{equation}
    \rightarrow \hat{n}(k, t)=\hat{n}\left(k, t_0\right) \exp \left(-D k^2\left(t-t_0\right)\right)=N_0 \exp \left(i k x_0\right) \exp \left(-D k^2\left(t-t_0\right)\right)
\end{equation}
so with the inverse Fourier transform (completing the square and using that a Gaussian distribution is normed)
\begin{equation}
    \text{a Gaussian } n(x, t)=\frac{N_0}{\sqrt{4 \pi D\left(t-t_0\right)}} \exp \left(-\frac{\left(x-x_0\right)^2}{4 D\left(t-t_0\right)}\right)
\end{equation}
\greenbox{\textbf{Conservation:} As the Gaussian is normalized, $n_0$ stays constant irrespective of time implying particle 
number conservation (or energy conservation for the diffusion of heat (heat conduction))}
\redbox{\textbf{Problem of particle propagation: } As the solution is a Gaussian with infinite wings, 
there must have been infinitely fast particle transport starting from the initial situation - unphysical.
\textbf{Reason - approximation in diffusion equation}: Our diffusion equation is based on a random walk with well-behaved 
steps but in finding the diffusion equation we have done a truncated Taylor expansion, leaving away $\mathcal{O}(\Delta x^4)$. 
In numerical schemes we will use limited flux.}

\redbox{\textbf{Infinite Signal Speeds: } In contrast to advection with a given signal speed the diffusion equation and also
Poisson equation (e.g. for Gravity) have infinite domains of influence - an event at $(\vec{x},t)$ depends on the fill domain.
In the case of diffusion we will later introduce limited flux aka tempered diffusion.}

\subsection{Numerical solutions}
Let us change the notation $n \rightarrow u$ to be consistent with previously introduced schemes.
We indicate with the convention $u_{\text{space}}^{\text{(time)}}$.

\bluebox{In the following we discuss multiple possible discretizations.}

The general grid in space and time is illustrated in figure \ref{fig:diff_grid}.

\begin{figure}[h]
    \centering
    \includesvg[width=0.8\textwidth]{figures/diff_grid.svg}
    \caption{Grid in space and time for the diffusion equation.}
    \label{fig:diff_grid}
\end{figure}

\subsubsection{Forward in time, central in space}
\subsubsubsection{Discretized diffusion equation}
Based on the central approximation of $\partial_x^2 u$ (from Taylor in 
both directions, see eq. \ref{eq:2nd_disc}) at the current
time $j$ and forward difference in time, we get
\begin{equation}
    \frac{u_i^{(j+1)} - u_i^{(j)}}{\Delta t} = D \frac{u_{i+1}^{(j)}-2 u_{i}^{(j)}+u_{i-1}^{(j)}}{\Delta x^2}
\end{equation}

\subsubsubsection{Explicit scheme for performing a time step}
Defining $\alpha = \frac{D \Delta t}{\Delta x^2}$, we get
\begin{equation}
    u_i^{(j+1)} = u_i^{(j)} + \alpha \left( u_{i+1}^{(j)} - 2 u_i^{(j)} + u_{i-1}^{(j)} \right)
\end{equation}
an explicit scheme for forwarding. The information flow is illustrated in 
figure \ref{fig:diff_explicit}.

\begin{figure}[h]
    \centering
    \includesvg[width=0.8\textwidth]{figures/diff_grid_central_forward.svg}
    \caption{Information flow for the central in space, forward in time scheme.}
    \label{fig:diff_explicit}
\end{figure}

\subsubsubsection{Stability of the central in space, forward in time scheme for diffusion}
Let us make a simple stability analysis considering $u_{i+1}^{(j)} = u_{i-1}^{(j)} = 0$.
In such a scenario it does not make sense for there to be a sign change from
$u_i^{(j)}$ to $u_i^{(j+1)}$. This requires (just plug conditions in)
\begin{equation}
    \alpha \leq \frac{1}{2} \leftrightarrow \Delta t \leq \frac{1}{2} \frac{\Delta x^2}{D}
\end{equation}
\bluebox{\textbf{Physical intuition and connection to CFL: } In diffusion, density spreads with $\Delta x = \sqrt{2 D \Delta t}$,
so when we impose a CFL criterion - we only use information from the left and right grid points next to the grid point of
interest, so $\Delta t$ must not be so large, that further information would in reality be necessary - we get
the same criterion as above.}
\redbox{\textbf{Problem of scaling the resolution:} If we want to double the spatial resolution, we must quadruple 
the resolution in time, i. e. in total the 1D simulation is 8x more expensive (in 3D double spatial resolution means 
8x more points in a certain volume so 32x more cost in total).}

\subsubsection{Backward in time, central in space}
\subsubsubsection{Discretized implicit scheme}
We now evaluate the spatial derivative at time $j+1$ instead of $j$.
\begin{equation}
    \frac{u_i^{(j+1)} - u_i^{(j)}}{\Delta t} = D \frac{u_{i+1}^{(j+1)}-2 u_{i}^{(j+1)}+u_{i-1}^{(j+1)}}{\Delta x^2}
\end{equation}
This can be rewritten as
\begin{equation}
     -\alpha u_{i+1}^{(j+1)} + (1+2\alpha) u_i^{(j+1)} - \alpha u_{i-1}^{(j+1)} = u_i^{(j)}
\end{equation}
- a linear system of equations for $u_i^{(j+1)}$.
\subsubsubsection{Matrix equation}
The exact matrix depends on the boundary conditions we choose. For periodic boundary conditions\footnote{The point before the 
first point is the last point and the point after the last point is the first point.} we get
\begin{equation}
    \begin{gathered}
        \mat{A} \vec{u}^{(j+1)} = \vec{u}^{(j)}, \quad \vec{u}_i^{(j)} = \left( \begin{array}{c}
            u_1^{(j)} \\
            u_2^{(j)} \\
            \vdots \\
            u_{N}^{(j)}
        \end{array} \right) \\
        \mat{A} = \begin{pmatrix}
            1+2\alpha & -\alpha & 0 & \dots &  & 0 & -\alpha \\
            -\alpha & 1+2\alpha & -\alpha & 0 & \dots &  &  \\
            0 & -\alpha & 1+2\alpha & -\alpha & 0 & \dots &  \\
             &  & \ddots & \ddots & \ddots &  &  \\
            &  &  & \ddots & \ddots & \ddots &  \\
            & &  & & \ddots  & \ddots  & \ddots  \\
            -\alpha & 0 & \dots &  & 0 & -\alpha & 1+2\alpha
        \end{pmatrix}
    \end{gathered}
\end{equation}
which we can solve for $\vec{u}^{(j+1)} = \mat{A}^{-1} \vec{u}^{(j)}$.
\problem{This is unconditionally stable but only first order in time $\mathcal{O}(\Delta x^2, \Delta t)$.}

\subsubsection{Crank-Nicolson method}
\idea{Evaluate the spatial derivative at $j+\frac{1}{2}$, constructed as an average of $j$ and $j+1$.}
\begin{equation}
    \frac{u_i^{(j+1)}-u_i^{(j)}}{\Delta t}=D \frac{u_{i+1}^{\left(j+\frac{1}{2}\right)}-2 u_i^{\left(j+\frac{1}{2}\right)}+u_{i-1}^{\left(j+\frac{1}{2}\right)}}{\Delta x^2}, \quad \text { use avg. } u_i^{\left(j+\frac{1}{2}\right)}=\frac{u_i^{(j+1)}+u_i^{(j)}}{2}
\end{equation}
yielding
\begin{equation}
    -\alpha u_{i+1}^{(j+1)}+(1+2 \alpha) u_i^{(j+1)}-\alpha u_{i-1}^{(j+1)}=\alpha u_{i+1}^{(j)}+(1-2 \alpha) u_i^{(j)}+\alpha u_{i-1}^{(j)}
\end{equation}
So again a matrix problem over the whole spatial domain
\begin{equation}
    \mat{A} \vec{u}^{(j+1)} = \vec{b}, \quad \mat{A} \text{ as before}, \quad b_i = \alpha u_{i+1}^{(j)}+(1-2 \alpha) u_i^{(j)}+\alpha u_{i-1}^{(j)}
\end{equation}
where again we can solve for $\vec{u}^{(j+1)} = \mat{A}^{-1} \vec{b}$. Note for non-periodic boundary conditions
e.g. Dirichlet boundaries (fixed), $\mat{A}$ is tridiagonal, so very easy to solve.
\greenbox{Advantage of Crank-Nicolson: $\mathcal{O}( \Delta x^2, \Delta t^2)$ (2nd order in space and time) and unconditionally stable.}

\subsection{Flux-limited diffusion (\textit{tempered})}
As of the approximation in the diffusion equation infinitely large signal speeds 
(unphysical) can result. Instead of including higher order terms in the Taylor expansion 
($\rightarrow$ hyperbolic system with finite information speed) \textcolor{blue1}{we limit the speed by hand 
using a limited diffusion flux}.
\idea{Express the flux as a product of density and velocity and limit this velocity akin to Special Relativity (based on a momentum consideration), e.g. to the speed of sound.}
Let us start with the diffusion equation
\begin{equation}
    \partial_t u = - \partial_x J, \quad J = - D \partial_x u
\end{equation}
We make the Ansatz
\begin{equation}
    J(x,t) = u(x,t) \cdot v, \quad \text{characteristic velocity } v
\end{equation}
and define a \textit{negative specific diffusion momentum}, really just the negative of the
not-yet-limited velocity $v$.
\begin{equation}
    -v = -\frac{J}{u} = \frac{D\partial_x u}{u} \equiv R
\end{equation}
Now we ask ourselves: What if $R$ was a relativistic momentum? What velocity $\tilde{v}$ would have brought forth 
this relativistic momentum will
naturally be limited to be smaller than $c$. So drawing from $p = \gamma m v, v<c$ we write
\begin{equation}
    R = - \frac{\tilde{v}}{\sqrt{1-\frac{\tilde{v}^2}{c^2}}} \quad \leftrightarrow \quad \tilde{v} = -\frac{R}{\sqrt{1+\frac{R^2}{c^2}}}
\end{equation}
so we use the altered flux
\begin{equation}
    \tilde{J} = u \tilde{v}, \quad \partial_t u = \partial_x \tilde{J}
\end{equation}

\subsection{Diffusion in three dimensions}
In 3D, we have
\begin{equation}
    \partial_t u(\vec{x}, t) = -\vec{\nabla} \cdot \vec{J}(\vec{x},t), \quad \text{flux vector } \vec{J}(\vec{x},t) = - \mat{D}\cdot \vec{\nabla} u(\vec{x},t), \quad \text{diffusion matrix } \mat{D}
\end{equation}
For a magnetized plasma, movement is mostly 
along the magnetic field lines (diffusion 
along the lines is dampened by collisions, 
diffusion perpendicular can only exist as of 
collisions).

\begin{equation}
    \begin{gathered}
        \vec{B}=B \vec{\hat{e}}_Z \rightarrow D=\left(\begin{array}{ccc}
            D_{\perp} & 0 & 0 \\
            0 & D_{\perp} & 0 \\
            0 & 0 & D_{\|}
            \end{array}\right) \\
            \text { typically in MHD: } D_{\perp} \ll D_{\|} \rightarrow \partial_t \vec{u}(\vec{x}, t)=\vec{\nabla} \cdot D_{\|} \vec{\hat{e}}_z\left(\vec{\hat{e}}_z \vec{\nabla} \vec{u}(\vec{x}, t)\right)
    \end{gathered}
\end{equation}

\pagebreak