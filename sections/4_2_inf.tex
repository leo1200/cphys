\section{Statistical Inference}
\thispagestyle{plain}

\subsection{Introduction to Inference}
We assume the black box of nature
to be some model\footnote{So we are more in the data modeling culture here.} $F_\theta$ where from a sample
\begin{equation}
    X_N = \left\{ x_1,\dots,x_N \right\}, \quad \text{sample size } N
\end{equation}
we would like to draw conclusions on a population
\begin{equation}
    \text{population size } K \gg N
\end{equation}
and estimate parameters $\hat{\theta}$, see figure \ref{fig:inference}.

\bluebox{\textbf{Goal in statistical inference:} Find out about the underlying principles that generated the data by formulating
models and inferring their parameters.}

\begin{figure}[!htb]
 \centering
 \includesvg[width=0.8\textwidth]{figures/inference.svg}\hfill
 \caption{Statistical inference}
 \label{fig:inference}
\end{figure}

Each sample from the population has the probability

\begin{equation}
    p(x_N) = \frac{1}{\left( \begin{array}{c}
        K \\ N
    \end{array} \right)}
\end{equation}

\subsubsection{Statistics}
Any quantity computed from values in a sample is called statistic
\begin{equation}
    T(x), \quad \text{e.g. } \bar{x}_N = \frac{1}{N} \sum_{i=1}^N x_i, s_N^2 = \frac{1}{N} \sum_{i=1}^N (x_i - \bar{x}_N)^2
\end{equation}
which is done
\begin{itemize}
    \item for estimating a population parameter (e.g. the mean)
    \item or evaluating a hypothesis (does the calculated statistic on the sample conform to its assumed statistic under the hypothesis?)
\end{itemize}
We want those statistics to be maximally informative (sufficient) for the population parameters,
and in the limit of the sample size $\rightarrow \infty$ to equal the population parameters, e.g.
\begin{equation}
    \lim_{N\rightarrow \infty} \bar{x} = E[x] =: \mu, \quad \lim_{N\rightarrow \infty} s_n^2 = \Var(x)
\end{equation}
more formally, the \textit{weak law of large numbers} state for the mean
\begin{equation}
    \bar{x}_N \overset{P}{\rightarrow} \mu \text{ meaning } \forall \epsilon > 0: \lim_{N \rightarrow \infty} p(|\bar{x}_N - \mu | > \epsilon) = 0, \quad \text{probability } p
\end{equation}
and the \textit{strong law of large numbers}
\begin{equation}
    \bar{x}_N \overset{a.s.}{\rightarrow} \mu \text{ meaning } p(\lim_{N\rightarrow \infty} \bar{x}_N = \mu) = 1
\end{equation}
where a. s. stands for almost surely.

\subsubsection{General Questions we ask ourselves in statistical modeling}
Consider we want to carry out an experiment, e.g. we measure lifetimes for a certain disease.
\begin{itemize}
    \item What model can we use for our data? Or should we better take an \textit{algorithmic approach} in the first place?
    \item What are good statistics, i.e. what quantities should we compute from our sample?
    \item How large should a sample be?
    \item What is a good estimate for a population parameter $\theta$? How can we obtain an estimate $\hat{\theta}$?
    \item How accurate do we assume this estimate to be?
\end{itemize}

\subsection{Examples of Statistical Models - baseline of statistical modeling}
We formulate a supposedly underlying model of our data with population parameters (denoted in Greek letters)
where from test statistics (Roman letters) we want to infer parameter estimates (Greek letters with hats)
or test hypothesis on them.

How can a statistical model look like?

\subsubsection{General form of the models}
A model has the general form
\begin{equation}
    \text{response } y_i(\text{features } \vec{x}_i) = \text{structural part}_\theta (\vec{x}_i) + \text{random part} \epsilon_i
\end{equation}
where the random part - under the assumption of many small additive errors - is usually modelled by
\begin{equation}
    \epsilon_i \sim \mathcal{N}(0,\sigma^2)
\end{equation}

\subsubsection{Example I: One- and two-factor univariate analysis; discrete RVs\skipthis}
In the following we consider simple models where discrete characteristics (e.g. a disease is present)
are modeled by constant terms across samples.
\paragraph*{One-factor:} The virus load of patient $i$ medicated with drug $j$, $x_{ij}$ could be described by
\begin{equation}
    \begin{gathered}
        x_{ij} = \mu + \tau_j + \epsilon_ij, \quad \text{random part } \epsilon_{ij} \sim \mathcal{N}(0,\sigma^2) \text{ i.i.d. so } \forall i,j \neq k,l: E[\epsilon_{ij}\epsilon_{ij}] = 0 \\
        \text{mean contribution } \mu, \quad \text{drug specific effect } \tau_j \text{ (assume no patient specificity)}
    \end{gathered}
\end{equation}
$\epsilon_{ij} \sim \mathcal{N}(0,\sigma^2)$ is a distributional assumption.
\paragraph*{Two-factor:} Adding e.g. a gender ($k$) effect $\beta_k$ and cross effect $\alpha \beta_{jk}$ (present
when the gender effect varies between drugs)
\begin{equation}
    x_{ijk} = \mu + \alpha_j + \beta_k + \alpha \beta_{jk} + \epsilon_{ijk}
\end{equation}

\subsubsection{Example II: Linear regression model}
Consider a dataset
\begin{equation}
    D = \{\text{features, response}\}_{i=1}^N, \quad i = 1,\dots,N
\end{equation}
with a linear model being
\begin{equation}
    \begin{gathered}    
        y(\vec{x}) = \vec{\beta}^* \cdot \vec{x} + \epsilon(x), \quad \text{true parameter } \vec{\beta}^* = \left( \begin{array}{c} \beta_0 \\ \beta_1 \\ \vdots \\ \beta_p \end{array} \right) \\
        \text{augmented independent variables} \left( \begin{array}{c} 1 \\ x_1 \\ \vdots \\ x_p \end{array} \right)
    \end{gathered}
\end{equation}
so if there is only one feature $x_i$, we have
\begin{equation}
    y_i = \beta_0 + \beta_1 x_i + \dots + \epsilon_i
\end{equation}

\note{A linear model is linear in the parameters, not necessarily the independent variables. For instance
\begin{equation}
    y_i = \beta_0 + \beta_1 x_i + \beta_2 x_i^2 + \dots + \epsilon_i
\end{equation}
is a linear model (polynomial regression).
}

\note{\textbf{How to check if a linear model breaks down?:} Often plots of the residuals are done. If they show a trend,
i.e. they are not i.i.d. with zero mean, our model is probably not applicable.}

\subsubsection{Example III: Generalized linear model}
Here we generally assume the data was brought forth from a distribution from the exponential family
\begin{equation}
    \begin{multlined}
        p(y | \theta) = \exp{\left( \frac{y\theta - b(\theta)}{a(\phi)} + c(y,\phi) \right)},  \\
        \text{with real functions } a,b,c, \quad \text{ and parameters } \theta, \phi
    \end{multlined}
\end{equation}
which mean we model.
\begin{equation}
    E[y_i\mid x_i] = \mu = g^{-1}(\eta_i) = \partial_\theta b, \quad \text{linear predictor } \eta_i = \vec{\beta}^T \vec{x}_i 
\end{equation}
\note{The notation is slightly different from before. For a \textit{canonical link} we have $\theta = \eta$ as before.}

\subsubsection{Example IV: Autoregressive model}
In the model for a time-series, e.g. the development
\begin{equation}
    x_t = \alpha x_{t-1} + \epsilon_t
\end{equation}
the $x_t$ are of course on i.i.d.

\subsection{Estimation of Model Parameters}
Consider we now have our model $F_\theta$. For instance, 
we could assume a linear model for the yearly temperature mean
on earth. We might be interested in
\begin{itemize}
    \item make point estimates for the parameters, e.g. the temperature increases by $0.06\K$ per decade
    \item the estimated deviation of our estimator over many samples (see figure \ref{fig:dif_sam_dif_est}) (the \textit{sampling distribution}), called standard error, so \textbf{an error approximation for our estimator}
    \item based on the standard error and distributional assumptions on the estimator, interval estimates for the parameters,
    e.g. we are $95\%$ certain that the temperature increases by $0.05\K$ to $0.07\K$ per decade
    \item hypothesis tests about model parameters (or other statistics from the data), for instance to test if there even is a linear trend
    in the mean temperature, e.g. with \textit{null-hypothesis} (no difference, devils advocate): there is no global warming, $H_0:\mu = 0, H_1:\mu\neq 0$
\end{itemize}

\begin{figure}[!htb]
 \centering
 \includesvg[width=1.0\textwidth]{figures/dif_sam_dif_est.svg}\hfill
 \caption{Different sample, different estimate - the estimator itself is a random variable}
 \label{fig:dif_sam_dif_est}
\end{figure}

We will first consider what a good estimator is and then discuss strategies for estimation.

\subsubsection{Properties of estimators}
What makes a good estimator?

\subsubsubsection{Bias of an estimator}
If the expected value of our estimator over many samples
\begin{equation}
    E[\hat{\theta}_N] = \theta^* + c, \quad \text{true parameter } \theta^*, \text{bias } c
\end{equation}
does not equal the population parameter, it is called biased, for $c = 0$ unbiased.
Good estimators are unbiased.

\subsubsubsection{Consistency of an estimator}
In the limit of an infinitely large sample, the
estimator should \underline{in probability} converge to the
true parameters
\begin{equation}
    \forall \epsilon: \lim_{N \rightarrow \infty} p(|\hat{\theta}_N - \theta^*| < \epsilon) = 1
\end{equation}

\subsubsubsection{Examples for bias and consistency at the hand of the mean}
\textbf{Aim:} We want to estimate the expectation $E[X]$ from an i.i.d. sample
$\{x_i\}_{i=1}^N$. For the total population, we know
\begin{equation}
    E[X] = \mu = \frac{1}{K} \sum_{i=1}^{K} x_i, \quad \text{size of the total population } K
\end{equation}
We consider different estimators in table \ref{tab:bias_consist}.

\begin{table}
    \centering
    \begin{tabular}{|p{0.3\linewidth}|p{0.3\linewidth}|p{0.3\linewidth}|}
        \hline \textcolor{blue1}{Estimator} & \textcolor{blue1}{Bias} & \textcolor{blue1}{Consistency} \\
        \hline
        $$\hat{\mu}_I=x_i, \quad \text{random } i$$ 
        &
        Unbiased, as $E\left[\hat{\mu}_I\right]=\mu$. &
        Inconsistent, $N \rightarrow \infty$ as
        sample size makes no
        difference as we pick a 
        random value. \\
        \hline
        $$\hat{\mu}_{I I}=\frac{1}{N} \sum_{i=1}^N x_i+\frac{1}{N}$$ &
        Biased as
        $$E\left[\hat{\mu}_I\right]=\mu+E\left[\frac{1}{N}\right]$$
        &
        Consistent as for $N \rightarrow \infty$
        $\frac{1}{N} \rightarrow 0$ and thus $\hat{\mu}_{I I} \rightarrow \mu$
        \\
        \hline
        $$\hat{\mu}_{I I I}=\frac{1}{N} \sum_{i=1}^N x_i$$ & Unbiased & Consistent \\
        \hline
    \end{tabular}
    \caption{Bias and consistency at the hand of estimators for the mean}
    \label{tab:bias_consist}
\end{table}

\subsubsubsection{The naive sample variance estimation is a biased but consistent estimator}
The population variance is
\begin{equation}
    \sigma^2=\frac{1}{K} \sum_{i=1}^K\left(x_i-\bar{x}\right)^2
\end{equation}
The analogue sample estimator for the population variance
\begin{equation}
    s_x^2=\frac{1}{N} \sum_{i=1}^N\left(x_i-\bar{x}\right)^2
\end{equation}
however is biased as of (proof later)

\begin{equation}
    E\left[s_x^2\right]=\frac{N-1}{N} \sigma^2
\end{equation}

so we use the \textit{Bessel-correction}
\begin{equation}
    \hat{\sigma}_{\text {unbiased }}^2=\frac{1}{N-1} \sum_{i=1}^N\left(x_i-\bar{x}\right)^2
\end{equation}

\greenbox{Our upscaled correction $\hat{\sigma}^2_\text{unbiased}$ takes into account that $\bar{x}$ is the sample estimate
not the population parameter from which the values from the sample naturally deviate less than from the true mean.}

\yellowbox{\textbf{While $\hat{\sigma}_{\text {unbiased }}^2$ is unbiased, $\hat{\sigma} = \sqrt{\hat{\sigma}_{\text {unbiased }}^2}$ generally is not.}
This can be followed from Jenssens inequality
\begin{equation}
    \begin{gathered}
        X \text{ a random variable, } \phi \text{ a convex function } \quad \rightarrow \quad \phi(E[X]) \leq E[\phi(X)] \\
        \phi \text{ a concave function } \quad \rightarrow \quad \phi(E[X]) \geq E[\phi(X)]
    \end{gathered}
\end{equation}
by
\begin{equation}
    \begin{gathered}
        \text { assume we use } \hat{\sigma}=\sqrt{\frac{1}{N-1} s_x^2} \text { as our estimator, then } \\
        E[\hat{\sigma}]=E\left[\sqrt{\frac{1}{N-1} s_x^2}\right] \underset{\text { Jennsens ineq. }}{\leq} \sqrt{E\left[\frac{1}{N-1} s_x^2\right]}=\sigma
    \end{gathered}
\end{equation}
}

\greybox{\textbf{Intuition for Jenssens inequality on the expectation:} A convex function will stretch out the 
right tail of the distribution \textbf{increasingly along the x-axis}. Therefore the expectation of the new distribution
we got by applying $\phi$ to the random variable will be shifted to higher values than the expectation of the original
distribution itself. This is illustrated in figure \ref{fig:jenssenX}.}

\begin{figure}[!htb]
 \centering
 \includesvg[width=0.8\textwidth]{figures/jenssenX.svg}\hfill
 \caption{Illustration of Jenssens inequality on the expectation. As the right tail is drawn increasingly further out for
 a convex function, the expectation of the transformed distribution is higher than the transformation of the mean 
 of the original distribution. The opposite holds for concave functions.}
 \label{fig:jenssenX}
\end{figure}

\subsubsubsection{Sampling distribution}
Consider we repeatedly draw samples of size $N$ and estimate $\hat{\theta}_N$. The distribution of
those estimates is called sampling distribution, denoted $F_N(\hat{\theta})$.

\subsubsubsection{Standard error of an estimator}
Consider an estimator $\hat{\theta}_N$ with sampling distribution $F_N(\hat{\theta})$. We then call
the standard deviation of the sampling distribution standard error.

\begin{equation}
    \text{variance in } \hat{\theta}_N \quad \rightarrow \quad \operatorname{SE}_{\hat{\theta}}(N) := \sqrt{E[(\hat{\theta}_N - E[\hat{\theta}_N])^2]}
\end{equation}

we can also define the \textbf{mean squared error}, accounting for bias and variance in $\hat{\theta}_N$
\begin{equation}
    \operatorname{MSE}(\hat{\theta}_N) = \sqrt{E\left[ (\hat{\theta}_N - \theta^*)^2 \right]}, \quad \text{true parameter } \theta^*
\end{equation}

where for an unbiased estimator ($E[\hat{\theta}_N] = \theta^*$) so $\operatorname{SE}_{\hat{\theta}}(N) = \operatorname{MSE}(\hat{\theta}_N)$.

\paragraph*{Standard error of the mean}
\begin{equation}
    \begin{gathered}
        S E M=S E_{\widehat{\mu}}(N)=\sqrt{\operatorname{Var}\left(\bar{x}_N\right)}=\sqrt{\operatorname{Var}\left(\frac{1}{N} \sum_{i=1}^N x_i\right)} \\
        \underset{\substack{\operatorname{Var}(a X+b)=a^2 \operatorname{Var}(X) \\
        \sum_i \operatorname{Var}\left(X_i\right)} \underset{X_i \text{ independent}}{=} \sum_i \Var(X_i)}{=} \frac{1}{N} \sqrt{\sum_{i=1}^N \operatorname{Var}\left(x_i\right)}=\frac{1}{N} \sqrt{N \sigma}=\frac{\sigma}{\sqrt{N}}
    \end{gathered}     
\end{equation}

\greenbox{\textbf{General estimation of standard errors by bootstrapping}: If for the statistic of interest there is no explicit formula
for the standard error and we only have one sample at hand, we can estimate the standard error by the standard deviation
of resampled samples from our sample (by choosing randomly from our sample with replacement). More on this later :)}

\subsubsubsection{Sufficient statistic}
\bluebox{\textbf{Aim:} We want to crunch down the $N$ data points in a sample, to one or a few statistics, 
which possibly contain all information available in the sample about the population parameter(s).}
A statistic or set of statistics $t(\vec{X})$ is sufficient if it contains all information there is in a sample
about a population parameter $\theta$ (/ population parameters $\vec{\theta}$), so
\begin{equation}
   p(\vec{X} | t(\vec{X}), \theta) = p(\vec{X} | t(\vec{X}))
\end{equation}
so \textbf{the likelihood of the sample is independent of the population parameter(s) given the sufficient statistic(s)},
no matter what $\theta$ is, given $t(\vec{X})$, the sample is equally likely. Given $t$ we cannot obtain more
information about $\theta$ from the sample, e.g. the total number of succeses in a sample of Bernoulli trials is sufficient
for the probability of success $\mu$.

\paragraph*{Minimal sufficiency of a set of statistics} The smallest set of sufficient statistics, the minimum number of estimators
carrying all information about the population, is called \textbf{minimally sufficient}, e.g. for a Gaussian these
are the sample mean $\mu$ and variance $\frac{N}{N-1} s_x^2$ (completely specifying the normal distribution). There
usually many possible sets of sufficient statistics but only one minimally sufficient one which the others
are functions of.

\subsubsubsection{Efficiency of an estimator}
Consider $\hat{\theta}_\text{opt}$ to be the estimator with the smallest variance (smallest standard error).
The efficiency of any other estimator $\hat{\theta}_k$ is then
\begin{equation}
    \operatorname{Eff}_{\hat{\theta}_k} = \frac{\operatorname{Var}(\hat{\theta}_\text{opt})}{\operatorname{Var}(\hat{\theta}_k)} \in [0,1]
\end{equation}
\note{The maximum likelihood estimator (the estimator under which the observed data is most probable) is an efficient estimator.}

\subsubsubsection{Precision of an estimator}
We can define the precision of an estimator as
\begin{equation}
    \operatorname{Prec}(\hat{\theta}_N) = \frac{1}{\operatorname{Var}(\hat{\theta}_N)}
\end{equation}
- the less the estimator varies over many samples, the more precise it is. For an
unbiased estimator, the precision is fundamentally limited by the \textit{Fisher information}\footnote{
    The partial derivative of the logarithm of the likelihood $p(\vec{X} | \theta)$ with respect to $\theta$ is called score.
    The Fisher information is the variance of the score, so
    \begin{equation}
        \mathcal{I}(\theta) = E\left[ \left( \frac{\partial}{\partial \theta} \log p(\vec{X} | \theta) \right)^2 \mid \theta \right] = \int \left( \frac{\partial}{\partial \theta} \log p(\vec{X} | \theta) \right)^2 p(\vec{X} | \theta) d\vec{X}
    \end{equation}
}, so
\begin{equation}
    \Var(\hat{\theta}_N) \geq \frac{1}{\mathcal{I}(\theta)}
\end{equation}
(Cramer-Rao bound).

\subsubsubsection{What characterizes a good estimator?}
A good estimator is
\begin{itemize}
    \item unbiased (its expected value is the population one)
    \item consistent (in the limit of a large sample we get the population parameter)
    \item efficient, so low standard error (it should not vary a lot between different samples)
    \item (minimally) sufficient (it should extract all information from the sample with regards to the population parameter of interest)
\end{itemize}

\subsubsection{Approaches to Statistical Parameter Estimation}
\bluebox{\textbf{Question:} What are the best model parameters for a given model which we can deduce from a sample?}

Principal ways of parameter estimation are
\begin{itemize}
    \item Least squared error estimation (LSE), minimize the (mean) squared error between the response predicted by the model under the estimated parameters and the sample
    \item maximum likelihoos estimation (MLE), choose parameters making the sample most likely
    \item Bayesian inference (BI), estimate the whole posterior of the parameters $p(\vec{\theta}|\{\vec{x}_i\}_{i=1}^N)$
    \item Maximum a posteriori (MAP), take the parameters maximizing the posterior $p(\vec{\theta}|\{\vec{x}_i\}_{i=1}^N)$
\end{itemize}

\subsubsubsection{Least squared error}
For the general model
\begin{equation}
    y_i = f_\vec{\theta}(\vec{x}_i) + \epsilon_i
\end{equation}
in LSE, we calculate the sum of squares loss on the sample $\{ \vec{x}_i,y_i \}_{i=1}^N$ as
\begin{equation}
    \operatorname{SSQ}(\vec{\theta}) = \frac{1}{N} \sum_{i=1}^{N} (y_i - \hat{y}_i)^2, \quad \text{model estimate } \hat{y}_i = E\left[y_i \mid x_i\right]=E\left[f_{\vec{\theta}}\left(\vec{x}_i\right)\right]+\underbrace{E\left[\epsilon_i\right]}_{=0}=f_{\vec{\theta}}\left(\vec{x}_i\right)
\end{equation}
where here $\vec{\theta}$ denote some model parameters. As an estimate for the true parameters $\vec{\theta}^*$ we can use
\begin{equation}
    \vec{\hat{\theta}}_{LSE} = \argmin_{\vec{\theta}} \operatorname{SSQ}(\vec{\theta})
\end{equation}

\paragraph*{Least squared error estimation for linear regression} Let us write the sample $\{ \vec{x}_i,y_i \}_{i=1}^N$ in the form

\begin{equation}
    \vec{Y} \in \mathbb{R}^N, \quad \mat{X} \in \mathbb{R}^{N\times p}
\end{equation}

where $\vec{Y}$ is called response or measurement and $\mat{X}$ is the feature matrix (independent variables).

We can then write the regression problem as

\begin{equation}
    \begin{gathered}
        \vec{Y} = f_\vec{\beta}(\mat{X}) + \vec{\epsilon} = \mat{X}\vec{\beta} + \vec{\epsilon}, \quad \text{residuals } \vec{\epsilon} \\
        \vec{\beta}=\left(\begin{array}{c}
            \beta_0 \\
            \vdots \\
            \beta_p
            \end{array}\right), \quad \text { feature matrix is augemented by prepended column of } 1 \text{s} \\
            \vec{Y} \text { is a random variable by virtue of the residual } \vec{\epsilon}, \quad \epsilon_i \sim N\left(0, \sigma^2\right) \text { i.i.d. }
    \end{gathered}
\end{equation}

We can calculate the sum of squares residual (a scalar) as

\begin{equation}
    \operatorname{SSQ}(\vec{\beta})=\|\vec{Y}-\mat{X} \vec{\beta}\|_2^2=(\vec{Y}-\mat{X} \vec{\beta})(\vec{Y}-\mat{X} \vec{\beta})^T=\|\vec{\epsilon}\|_2^2
\end{equation}

which is nicely convex, so we can find a unique minimum by

\begin{equation}
    \begin{gathered}
    \frac{\partial \operatorname{SSQ}}{\partial \vec{\beta}}=-2 \mat{X}^T(\vec{Y}-\mat{X} \vec{\beta})_{!}=0 \\
    \rightarrow \mat{X}^T \vec{Y}=\mat{X}^T \mat{X} \vec{\beta} \rightarrow \vec{\hat{\beta}}=\left(\mat{X}^T \mat{X}\right)^{-1} \mat{X}^T \vec{Y} \text { (normal equation) }
    \end{gathered}
\end{equation}

\note{For this to apply $\mat{X}^T \mat{X}$ must be invertible, so $\mat{X}$ must have rank $p$. This is e.g. not the case for the overparametrized 
$p>N$. $\mat{X}^T \mat{X}$ measures covariance, assuming the columns of $\mat{X}$ to be centered, it is the sample
covariance matrix (without a scaling factor), $\mat{\Sigma}_{XX} = \frac{1}{N-1} \mat{X}^T \mat{X} \in \mathbb{R}^{p\times p}$.}

\greybox{\textbf{1D case:} For $y_i = \beta_0 + \beta_1 x_i + \epsilon_i$ one finds from $\partial_{\beta_0} \operatorname{SSQ} = 0$ and $\partial_{\beta_1} \operatorname{SSQ} = 0$
\begin{equation}
    \begin{gathered}
        \hat{\beta}_0=\bar{y}-\hat{\beta}_1 \bar{x} \text{ interset as expected}\\
        \beta_1=\frac{\frac{1}{N-1} \sum\left(x_i-\bar{x}\right)\left(y_i-\bar{y}\right)}{\frac{1}{N-1} \sum\left(x_i-\bar{x}\right)^2}=\frac{\widehat{\operatorname{Cov}}(x, y)}{\widehat{\operatorname{Var}}(x)}
    \end{gathered}
\end{equation}
}

\subsubsubsection{Maximum likelihood estimation (MLE)}
The likelihood of a sample $\{ \vec{x}_i \}_{i=1}^N$ of $N$ i.i.d. data points
under a model $F_\vec{\theta}$ which assigns each data point a probability $p(\vec{x}_i | \vec{\theta})$ is
\begin{equation}
    \mathcal{L}_X(\vec{\theta}) = \prod_{i=1}^N p(\vec{x}_i | \vec{\theta})
\end{equation}
\greenbox{For e.g. linear regression this model is hidden in the assumption of the residuals being i.i.d. Gaussian
with variance $\sigma^2$, where
\begin{equation}
    p(y_i|\vec{x}_i,\vec{\beta}) = \mathcal{N}(\vec{\beta}^T \vec{x}_i, \sigma^2)
\end{equation}
In the generalized linear setting, this will not be a normal distribution but a distribution from the exponential family
and the mean will be modelled by a link function (its inverse).
}
\greenbox{\textbf{Idea:} Model parameters under which the sample is likely are probably good.}

\begin{equation}
    \boxed{\vec{\hat{\theta}}_{MLE} = \argmax_{\vec{\theta}} \mathcal{L}_X(\vec{\theta})}
\end{equation}

\problem{The product of many small probabilities can numerically be problematic (underflow)}
\idea{We can alternatively maximize the log-likelihood or minimize the negative log-likelihood
\begin{equation}
    \mathcal{LL}_X(\vec{\theta}) = \log \mathcal{L}_X(\vec{\theta}) = \sum_{i=1}^N \log p(\vec{x}_i | \vec{\theta})
\end{equation}
so
\begin{equation}
    \boxed{\vec{\hat{\theta}}_{MLE} = \argmax_{\vec{\theta}} \mathcal{LL}_X(\vec{\theta})}
\end{equation}
}

\paragraph*{MLE for linear regression - same result as LSE} Again consider a dataset $\{ \vec{x}_i,y_i \}_{i=1}^N$
with a linear model $y_i = \vec{\beta}^T \vec{x}_i + \epsilon_i$ and i.i.d. residuals $\epsilon_i \sim \mathcal{N}(0,\sigma^2)$.
The likelihood of the sample under some model parameters $\vec{\beta}$ is then (using $y_i|\vec{x}_i,\vec{\beta} \sim \mathcal{N}(\vec{\beta}^T \vec{x}_i,\sigma^2)$)
\begin{equation}
    \begin{gathered}
    \mathcal{L}_X(\vec{\beta})=\prod_{i=1}^N p\left(y_i \mid \vec{x}_i,\vec{\beta}\right)=\prod_{i=1}^N \frac{1}{\sqrt{2 \pi} \sigma} \exp \left(-\frac{\left(y_i-\vec{\beta}^T \vec{x}_i\right)^2}{2 \sigma^2}\right) \\
    =\left(\frac{1}{\sqrt{2 \pi} \sigma}\right)^N \exp \left(-\frac{\sum_{i=1}^N\left(y_i-\vec{\beta}^T \vec{x}_i\right)^2}{2 \sigma^2}\right)=\left(\frac{1}{\sqrt{2 \pi} \sigma}\right)^N \exp \left(-\frac{S S Q(\vec{\beta})}{2 \sigma^2}\right) \\
    \vec{\hat{\beta}}_{M L E}=\underset{\vec{\beta}}{\operatorname{argmax}} \mathcal{L}(\vec{\beta}) \Leftrightarrow \vec{\hat{\beta}}_{M L E}=\underset{\vec{\beta}}{\operatorname{argmin}} S S Q(\vec{\beta})=\vec{\hat{\beta}}_{L S E}
    \end{gathered}
\end{equation}

\paragraph*{MLE for $\mu$ and $\sigma^2$ in a Gaussian model} Consider a sample $\{ x_i \}_{i=1}^N$ from a Gaussian distribution.
What could be estimates for the mean $\mu$ and variance $\sigma^2$?

From the univariate Gaussian distribution
\begin{equation}
    p(x_i | \mu, \sigma^2) = \frac{1}{\sqrt{2\pi\sigma^2}} \exp\left(-\frac{(x_i - \mu)^2}{2\sigma^2}\right)
\end{equation}
we calculate the log-likelihood $\mathcal{LL}_X(\mu,\sigma^2)$ and find
\begin{equation}
    \begin{gathered}
        \left. \partial_{\mu} \mathcal{LL}_X\right|_{\mu=\hat{\mu}_{MLE}} = 0 \quad \rightarrow \quad \hat{\mu}_{MLE} = \frac{1}{N} \sum_{i=1}^N x_i \\
        \left. \partial_{\sigma^2} \mathcal{LL}_X\right|_{\sigma^2=\hat{\sigma}^2_{MLE}} = 0 \quad \rightarrow \quad \hat{\sigma}^2_{MLE} = \frac{1}{N} \sum_{i=1}^N (x_i - \mu)^2
    \end{gathered}
\end{equation}
(where we have carried out the calculations as if the other parameter was known). For $\mu = \hat{\mu}_{MLE}$, $\hat{\sigma}^2_{MLE}$ is the
biased estimator for the population variance. Given the model estimate $\hat{y} = \vec{\beta}^T \vec{x}$, the \textbf{MLE for the variance
in linear regression} is
\begin{equation}
    \hat{\sigma}^2_{MLE} = \frac{1}{N} \sum_{i=1}^N (y_i - \hat{y}_i)^2 = \frac{1}{N} \sum_{i=1}^N \epsilon_i^2
\end{equation}

\subsubsubsection{Bayesian inference}
Consider we have measured a sample $\{ \vec{x}_i \}_{i=1}^N$ and 
want to estimate the model parameters $\vec{\theta}$.
\greenbox{By Bayes theorem we can calculate the full distribution of the parameters given the sample
and prior knowledge $p(\vec{\theta})$. \textbf{From there we can e.g. take the parameters most likely given the
data (MAP) or the expected value of the parameters, ...}. The prior (essentially a \textit{bias} or \textit{regularization})
is especially useful when the sample size is small - we can update our previous knowledge with new evidence - the sample.}

\begin{equation}
    \begin{gathered}
    \text { posterior }=p\left(\vec{\theta} \mid \vec{x}_1, \vec{x}_2, \ldots, \vec{x}_N\right)=\frac{p\left(\vec{x}_1, \vec{x}_2, \ldots, \vec{x}_N \mid \theta\right) p_{\vec{\alpha}}(\vec{\theta})}{h\left(\vec{x}_1, \vec{x}_2, \ldots, \vec{x}_N\right)}=\frac{\text { likelihood } \cdot \text { prior }}{\text { evidence }} \\
    =\frac{p\left(\vec{x}_1, \vec{x}_2, \ldots, \vec{x}_N \mid \vec{\theta}\right) p_{\vec{\alpha}}(\vec{\theta})}{\int p\left(\vec{x}_1, \vec{x}_2, \ldots, \vec{x}_N \mid \vec{\theta}^{\prime}\right) p_{\vec{\alpha}}(\vec{\theta}) d \vec{\theta}^{\prime}}, \quad \text { prior } p_{\vec{\alpha}} \text { with hyper-parameters } \alpha
    \end{gathered}
\end{equation}

\paragraph*{Updating our knowledge on the distribution of the model parameters given the data - one experiment at a time}
Consider we start out with the prior $p(\vec{\theta}) = p_{\vec{\alpha}}(\vec{\theta})$ with $\vec{\alpha}$ being the hyperparameters of
the prior.
\greybox{\textbf{Example:} For instance if our likelihood was a Binomial distribution (so the samples would consist of counts
of successes in $N$ trials) and we would model the only parameter of the Binomial distribution, the success probability $\mu$,
we could use a Beta prior, so that posterior and prior would be of the same distributional form (conjugacy). If we start out
without knowledge, we could use a uniform prior, so a Beta distribution with $\alpha = \beta = 1$.}

Consider $X$ is new data, the update to the prior is
\begin{equation}
    p^{(n)}(\vec{\theta} | X) = \frac{p(X | \vec{\theta}) p^{(n-1)}(\vec{\theta})}{h(X)}
\end{equation}

\paragraph*{MAP as a special case of Bayesian inference}
Maximizing the posterior yields the MAP estimate - the mode of the posterior distribution
\begin{equation}
    \vec{\hat{\theta}}_{\text{MAP}} = \argmax_{\vec{\theta}} p(\vec{\theta} | \vec{x}_1, \vec{x}_2, \ldots, \vec{x}_N) \underset{\text{Bayes}}{=} \argmax_{\vec{\theta}} \frac{p(\vec{x}_1, \vec{x}_2, \ldots, \vec{x}_N | \vec{\theta}) p_{\vec{\alpha}}(\vec{\theta})}{h(\vec{x}_1, \vec{x}_2, \ldots, \vec{x}_N)}
\end{equation}
which based on
\begin{itemize}
    \item i.i.d. assumption
    \item seeing that the evidence $h(\vec{x}_1, \vec{x}_2, \ldots, \vec{x}_N)$ is independent of $\vec{\theta}$
\end{itemize}
simplifies to
\begin{equation}
    \boxed{
    \begin{aligned}
    \vec{\hat{\theta}}_{M A P}&=\underset{\vec{\theta}}{\operatorname{argmax}} \frac{p_{\vec{\alpha}}(\vec{\theta}) \prod_{i=1}^N p\left(\vec{x}_i \mid \vec{\theta}\right)}{h\left(\vec{x}_1, \vec{x}_2, \ldots, \vec{x}_N\right)}\\
    &=\underset{\vec{\theta}}{\operatorname{argmax}}\left(p_{\vec{\alpha}}(\vec{\theta}) \prod_{i=1}^N p\left(\vec{x}_i \mid \vec{\theta}\right)\right) \\
    &=\underset{\vec{\theta}}{\operatorname{argmax}}\left(\log p_{\vec{\alpha}}(\vec{\theta})+\sum_{i=1}^N \log p\left(\vec{x}_i \mid \vec{\theta}\right)\right)
    \end{aligned}}
\end{equation}

\paragraph*{Pros and cons of Bayesian inference} See table \ref{tab:pros_cons_bayes}.
\begin{table}
    \centering
    \begin{tabular}{|p{0.45\linewidth}|p{0.45\linewidth}|}
        \hline \textcolor{green1}{Pros} & \textcolor{red1}{Cons} \\
        \hline
        \begin{itemize}
            \item \textbf{Using previous knowledge:} The prior is a \textit{bias}\tablefootnote{In the sense that the model is not prejudice-free but has previous knowledge.} regularizing the model parameters (in the sense that a low data setting will not bring forth unrealistic parameters just by chance)
            \item \textbf{Uncertainty quantification:} The full posterior distribution can be used to quantify the uncertainty in the model parameters (a flat posterior means we have little information about the parameters, a peaked one means we have a lot of information)
            \item \textbf{Hypothesis testing:} Based on the full posterior distribution, we can test hypotheses about the model parameters
            \item \textbf{Updating:} The prior can be updated with new evidence
        \end{itemize}
        &
        \begin{itemize}
            \item \textbf{Subjectivity of the prior:} Danger if prior is not reliable or incorrect\tablefootnote{This is a bias-variance trade-off: the stronger the bias, the less flexible the model but the less it will also vary for different samples.}
            \item \textbf{Computational cost:} The full posterior distribution is often hard to calculate, for instance the evidence integral is often intractable\footnote{Integrals involving the posterior can still be approximated without knowing the evidence using MCMC}
        \end{itemize}
        \\
        \hline
    \end{tabular}
    \caption{Pros and cons of Bayesian inference}
    \label{tab:pros_cons_bayes}
\end{table}

\yellowbox{\textbf{Note on the subjectivity of the prior:} In fact the entire model is subjective - both likelihood and prior. In the famous George
Box quote: \enquote{all models are wrong, but some are useful} - acknowledging that statistical models always fall short of the complexities of reality but can still be useful.}

\paragraph*{Bayesian inference for linear regression with normally distributed residuals} Consider we have a dataset
$\{ \vec{x}_i,y_i \}_{i=1}^N$ and a linear model
\begin{equation}
    y_i = \vec{\beta}^T \vec{x}_i + \epsilon_i
\end{equation}
with i.i.d. residuals $\epsilon_i \sim \mathcal{N}(0,\sigma^2)$.

\bluebox{\textbf{Question:} What is the posterior distribution of the model parameters $\vec{\beta}$ and $\sigma^2$ given the sample?}

Let us start with writing down the joint posterior of $\vec{\beta}$ and $\sigma^2$ (not denoting the fixed $\mat{X}$ (the colleciton of all $\vec{x}_i$) explicitly)
\begin{equation}
    \begin{aligned}
        \text{posterior}&=p(\vec{\beta},\sigma^2 | \vec{y}) \\
        &=\frac{p(\vec{y} | \vec{\beta},\sigma^2) \textcolor{green1}{p(\vec{\beta},\sigma^2)}}{h(\vec{y})} \\
        &=\frac{p(\vec{y} | \vec{\beta},\sigma^2) \textcolor{green1}{p(\sigma^2) \cdot p(\vec{\beta} \mid \sigma^2)}}{h(\vec{y})}
        &= \frac{\text{likelihood} \cdot \text{\textcolor{green1}{priors}}}{\text{evidence}}
    \end{aligned}
\end{equation}
The likelihood over $y$ at the $N$ fixed $\vec{x}_i$ is normal (with linearly modeled mean $\vec{\mu} = \mat{X}\vec{\beta} \in \mathbb{R}^N$) as we assumed normal residuals, so
\begin{equation}
    p\left(\vec{y} \mid \vec{\beta}, \sigma^2\right)=(2 \pi)^{-N / 2}\left|\sigma^2 \mat{1}\right|^{-1 / 2} \exp{\left(-\frac{1}{2}(\vec{y}-\mat{X} \vec{\beta})^T\left(\sigma^{-2} \mat{1}\right)(\vec{y}-\mat{X} \vec{\beta})\right)}, \quad \vec{y} \in \mathbb{R}^N
\end{equation}

\note{For a Gaussian likelihood (without $\sigma^2$ or $\mu$ being known a priori)
the prior $p(\vec{\beta} \mid \sigma^2)$ can be chosen to be Gaussian as well, and 
the prior $p(\sigma^2)$ can be chosen to be an inverse Gamma distribution (the total prior is
their product), so that the posterior is of the same distributional form as the prior (conjugacy).}

The prior for $\vec{\beta}$ is the Gaussian
\begin{equation}
    \text { Gaussian prior on } \vec{\beta} \text { : } p\left(\vec{\beta} \mid \sigma^2\right)=(2 \pi)^{-p / 2}\left|\sigma^{-2} \mat{\Lambda}_0\right|^{1 / 2} \exp{\left(-\frac{1}{2}\left(\vec{\beta}-\vec{\beta}_0\right)^T\left(\sigma^{-2} \mat{\Lambda}_0\right)\left(\vec{\beta}-\vec{\beta}_0\right)\right)}
\end{equation}
with
\begin{itemize}
    \item $\vec{\beta}_0$ being the mean of the prior of $\vec{\beta}$
    \item $\mat{\Lambda}_0$ being the precision matrix (inverse covariance matrix) of the prior of $\vec{\beta}$, so how certain we are that $\vec{\beta}_0$ is a good first guess (for the pre-factor note $\det(\mat{A}^{-1}) = (\det(\mat{A}))^{-1}$)
    \item $\sigma^2$ being only a parameter here, convenient for later calculations
\end{itemize}
so multiplied with the Gaussian likelihood, the posterior will be Gaussian as well.

The prior for $\sigma^2$ is the inverse Gamma distribution (i.e. $1\slash \sigma^2$ is Gamma distributed)\footnote{This can also be motivated based on the distribution
of the MLE estimate for $\sigma^2$ around the true $\sigma^2$ given by
\begin{equation}
    \hat{\sigma}^2_\text{MLE} = \frac{1}{N} \sum_{i=1}^N (y_i - \bar{y}_i)^2 \sim \frac{\sigma^2}{N} \chi^2_{N} \sim \frac{\sigma^2}{N} \Gamma\left(\frac{N}{2},2\right)
\end{equation}
so using for $X \sim \Gamma(\alpha,\beta)$, $cX \sim \Gamma(\alpha,c\beta), c > 0$, we can, by shady math, obtain (by switching to $\sigma$ being a latent variable not a fixed parameter)
\begin{equation}
    \frac{1}{\sigma^2} \sim \Gamma\left(\frac{N}{2},\frac{2}{N\hat{\sigma}^2_\text{MLE}}\right)
\end{equation}
}
\begin{equation}
    \text { Inverse - Gamma prior on } \sigma^2: \quad p\left(\sigma^2\right)=\frac{\kappa^\alpha \sigma^{2(-\alpha-1)}}{\Gamma(\alpha)} e^{-\kappa / \sigma^2}
\end{equation}
Multiplied with the Gaussian likelihood, the posterior will be inverse Gamma as well.

Where the priors have been chosen so that the posterior

\begin{equation}
    \begin{aligned}
        \text { posterior } &= \frac{\text { \textcolor{yellow1}{likelihood} } \cdot \text { \textcolor{green1}{priors} }}{\text { evidence }}\\
        &=\frac{\text { \textcolor{yellow1}{gaussian} } \cdot \text { \textcolor{green1}{gaussian} } \cdot \text { \textcolor{green1}{inverse gamma} }}{\text { evidence }} \\
        &=\text { gaussian } \cdot \text { inverse gamma }
    \end{aligned}
\end{equation}

is of the same distributional form as the prior (conjugacy) (the Gaussian of the likelihood and the Gaussian of the prior for $\vec{\beta}$ can be combined to one Gaussian).

\paragraph*{Update rules for the parameters from prior to posterior} Without proof, let us specify
\begin{itemize}
    \item the Gaussian distribution of $\vec{\beta}$ is updated by
    \begin{equation}
        \begin{gathered}
            \begin{aligned}
                \vec{\beta}_N &= \left(\mat{X}^T \mat{X} + \mat{\Lambda}_0\right)^{-1} \left(\mat{X}^T \vec{y} + \mat{\Lambda}_0 \vec{\beta}_0\right) \\
                &= \left(\mat{X}^T \mat{X} + \mat{\Lambda}_0\right)^{-1} \left(\mat{X}^T \mat{X} \vec{\hat{\beta}}_{\text{MLE}} + \mat{\Lambda}_0 \vec{\beta}_0\right)
            \end{aligned} \\
            \text{precision matrix } \mat{\Lambda}_N = \mat{X}^T \mat{X} + \mat{\Lambda}_0
        \end{gathered}
    \end{equation}
    a sort of weighted average (but not really as the as it is not a scalar equation) between the MLE estimate and the prior $\vec{\beta}_0$ weighted by the prior precision (all normalized by a sort of total covariance matrix)
    \item the inverse Gamma distribution of $\sigma^2$ is updated by
    \begin{equation}
        \alpha_N = \alpha_0 + \frac{N}{2}, \quad \kappa_N = \kappa_0 + \frac{1}{2} \left(\vec{y}^T \vec{y} + \vec{\beta}_0^T \mat{\Lambda}_0 \vec{\beta}_0 - \vec{\beta}_N^T \mat{\Lambda}_N \vec{\beta}_N\right)
    \end{equation}
    with the expectation of $\sigma^2$ being $E[\sigma^2 | \vec{y},\mat{X}] = \frac{\kappa_N}{\alpha_N - 1}$
\end{itemize}

\greenbox{For each subsequent update with new data $X$, the posterior of the parameters becomes the prior for the next update.}

\subsection{Hypothesis Testing}
\textbf{Aim of hypothesis testing\footnote{More specifically null hypothesis significance testing.}:} We want to decide 
whether some data sufficiently supports a \textcolor{pink1}{hypothesis}.
\pinkbox{\textbf{What is a hypothesis?}: A hypothesis is a statement about (a) population parameter(s) (not
about a sample). We usually formulate a devil's advocate null hypothesis $H_0$ (no difference,
as in $H_0:$ the drug we test had no effect, $H_0:$ there is no global warming, ...), under which we devise the
distribution of a statistic and under this distribution check how likely it is to observe
the statistic following from our sample. We are cautious and want to make sure that we have
to reject the no-difference case to make our statement (e.g. $H_1:$ the drug has an effect).}

\textbf{Example setting:} Consider we have a dataset $D=\{ x_i,y_i \}_{i=1}^N$ and assume a linear model
\begin{equation}
    \begin{gathered}
        y_i = \beta_0 + \beta_1 x_i + \epsilon_i, \quad \epsilon_i \sim \mathcal{N}(0,\sigma^2) \\
        \text{anthropogenic CO2 in the atmosphere } x_i, \quad \text{global temperature } y_i
    \end{gathered}
\end{equation}
with the devil's advocate $H_0: \beta_1 = 0$ (climate change is a hoax). Neglecting
$H_0$ means there is an impact of CO2 on temperature, a \textit{one-sided} can give
clues about the direction of the impact.

\problem{Null hypothesis significance testing is set up in a way that we crunch all our
data into a single \textbf{accept-reject decision} with a \textit{$p$-value}, where one who
lacks integrity might $p$-hack on. Bayesian statistics on the other hand gives a full
distribution over the parameters, but here a problem lies in the subjectivity of the prior.}

\subsubsection{Basic terms of statistical hypothesis testing}
Let us first introduce basic terms of hypothesis testing.
\subsubsubsection{Around the hypothesis}
See table \ref{tab:terms_hypothesis}.
\begin{table}
    \centering
    \begin{tabular}{|p{0.45\linewidth}|p{0.45\linewidth}|}
        \hline \textcolor{blue1}{Term} & \textcolor{blue1}{Short explanation} \\
        \hline
        Statistical hypothesis & Statement about the parameters of a model describing a population, e.g. the parameters in a linear model \\
        \hline
        Test statistic $T$ & Generally any value calculated from a sample \\
        \hline
        Simple hypothesis & Hypothesis specifying the population distribution completely \\
        \hline
        Composite hypothesis & Hypothesis not specifying the population distribution completely. \\
        \hline
        Null hypothesis $H_0$ & The statement tested in the statistical significance test, usually a statement of \textit{no-effect} or \textit{no difference}, e. g. COVID19 measures had no effect on the virus's reproduction rate. \\
        \hline
        Alternative hypothesis $H_1$ & Against the null hypothesis $H_0$ (mutually exclusive statements, often exhaustive) the alternative hypothesis is tested and accepted when the null hypothesis is rejected; usually consistent with the research hypothesis; e. g.: COVID19 measures had an effect on the virus's reproduction rate. \\
        \hline
    \end{tabular}
    \caption{Basic terms around the hypothesis}
    \label{tab:terms_hypothesis}
\end{table}

\subsubsubsection{Decision-making}
See table \ref{tab:dec_mak}.
\begin{table}
    \centering
    \begin{tabular}{|p{0.45\linewidth}|p{0.45\linewidth}|}
        \hline \textcolor{blue1}{Term} & \textcolor{blue1}{Short explanation} \\
        \hline
        Rejection region / ciritcal region & Set of values of the test statistic $t$ for which $H_0$ is rejected (this region is derived under $H_0$) \\
        \hline
        Acceptance region & Set of $t$-values where $H_0$ is accepted \\
        \hline
        Critical value & Boundary / boundaries of the acceptance region \\
        \hline
    \end{tabular}
    \caption{Basic terms around the decision-making in hypothesis tests}
    \label{tab:dec_mak}
\end{table}

\subsubsubsection{Validity of the decision}
See table \ref{tab:dec_mak}.
\begin{table}
    \centering
    \begin{tabular}{|p{0.45\linewidth}|p{0.45\linewidth}|}
        \hline \textcolor{blue1}{Term} & \textcolor{blue1}{Short explanation} \\
        \hline
        Significance level $\alpha$ & Probability of rejecting the null hypothesis $H_0$ given it is true (type 1 error). $\alpha$ is a
        parameter we choose and set the rejection region(s) accordingly. \\
        \hline
        $p$-value & Based on the distribution of the test statistic devised under $H_0$ it is the probability of observing a value
        at least as extreme as $t$ in our sample. $p$ is used to check if we are in the rejection region. \\
        \hline
        Statistical significance & If $p\leq \alpha$ we reject $H_0$ at the $\alpha$ significance level. So the chance
        that this sample was brought forth under the \textit{no-difference} assumption is less than $\alpha$.\\
        \hline
        Type 1 error, false positive (\textit{rejection is positive}) & Rejecting $H_0$ when it is true \\
        \hline
        Size of the test & Probability of a type 1 error, $\alpha$ \\
        \hline
        Power of a test, true positive & Probability of correctly rejecting $H_0$ when it is false, it is
        the probability of the rejection region under the \textit{true} distribution \\
        \hline
    \end{tabular}
    \caption{Basic terms around the decision-making in hypothesis tests}
    \label{tab:dec_mak}
\end{table}

\subsubsubsection{Statements on tests}
See table \ref{tab:test_stats}.
\begin{table}
    \centering
    \begin{tabular}{|p{0.45\linewidth}|p{0.45\linewidth}|}
        \hline \textcolor{blue1}{Term} & \textcolor{blue1}{Short explanation} \\
        \hline
        Exact test & The derivation of the distribution of the test statistic only depends on $H_0$ being
        true. So if $H_0$ is true, we will by chance reject on average \textit{exactly} the proportion $\alpha$ of tests. \\
        \hline
        Most powerful test & The test with the highest power for a given size $\alpha$ \\
        \hline
    \end{tabular}
    \caption{Basic terms around tests}
    \label{tab:dec_mak}
\end{table}

\subsubsection{Basic Null Hypothesis Significance Testing Procedure}
\idea{Based on a null hypothesis come up with a test statistic which distribution can be derived under $H_0$
and check if the observed statistic can reasonably be assumed to have been brought forth under $H_0$.}
\begin{enumerate}
    \item We want to test a statement about model parameters for a population. We therefore state the null-hypothesis
    $H_0$ and the alternative hypothesis $H_1$ in terms of these.
    \begin{equation}
        \begin{gathered}
            \text{e.g. the two-sided } H_1: \beta_1 \neq 0, \quad H_0: \beta_1 = 0 \\
            \text{e.g. the one-sided } H_1: \beta_1 > 0, \quad H_0: \beta_1 \leq 0
        \end{gathered}
    \end{equation}
    \note{For the one-sided test we in effect mostly also use the \textit{no-difference} $\quad H_0: \beta_1 = 0$ (the most
    \textit{extreme} version of $H_0: \beta_1 \leq 0$) to
    devise the test statistic and then based on \textit{where} we reject\footnote{On the left or right side of the distribution. The test statistic is 1-dimensional, while our statement might be concerned with multiple parameters.}
    can make a 1-sided statement.}
    \item Define a test statistic $T$ which can be calculated from our sample (and for which we can find a distribution)
    \begin{equation}
        \text{e.g. } t=\hat{\beta}_1^\text{MLE} \quad \text{for the linear model}
    \end{equation}
    \item Derive the distribution of the test statistic under $H_0$ (and possibly further assumptions) (or construct it empirically via \textit{bootstrapping})
    \begin{equation}
        \text{e.g. } t \sim \mathcal{N}(0,\sigma^2 (\mat{X}^T \mat{X})^{-1}) \quad \text{for the linear model, e.g. with } \beta_0 \text{ marginalized out}
    \end{equation}
    \item Define a significance level $\alpha$ (e.g. $\alpha = 0.05$)
    \item Partition the distribution of the test statistic $T$ into rejection and acceptance region. The rejection region is
    chosen to have probability $\alpha$ under $H_0$.
    \item Compute the observed value $t_{\text{obs}}$ of the test statistic $T$ from the sample
    \item If $t_{\text{obs}}$ is in the rejection region, reject $H_0$ at the $\alpha$ significance level, otherwise accept $H_0$.
\end{enumerate}
Steps 5 to 7 can alternatively be formulated as
\begin{enumerate}
    \item Compute the $p$-value of the observed test statistic $t_{\text{obs}}$, i.e. the probability of obtaining a sample
    at least as extreme as the observed sample under $H_0$.
    \item Reject $H_0$ in favor of $_1$ if $p \leq \alpha$.
\end{enumerate}

The testing procedure is illustrated in figure \ref{fig:hypothesis_testing}.

\begin{figure}
    \centering
    \includesvg[width=0.95\textwidth]{figures/hypothesis_testing.svg}
    \caption{Illustration of the hypothesis testing procedure}
    \label{fig:hypothesis_testing}
\end{figure}

\paragraph*{Rejection in a two-sided test:} In a two-sided test we construct equally likely
rejection regions on both sides of the distribution of the test statistic. Therefore, we reject
$H_0$ if
\begin{equation}
    p\left(t \geq t_{o b s} \mid H_0\right) \leq \frac{\alpha}{2}, \quad p\left(t \leq t_{o b s} \mid H_0\right) \leq \frac{\alpha}{2}
\end{equation}
Formally the rejection region is given by (let $F(t)$ be the CDP of $p(t)$)
\begin{equation}
    R=\left[-\infty, F^{-1}\left(\frac{\alpha}{2}\right)\right] \cup\left[F^{-1}\left(1-\frac{\alpha}{2}\right), \infty\right]
\end{equation}

\subsubsubsection{Type I and type II error}
\begin{itemize}
    \item \textbf{Type I error:} Rejecting $H_0$ when it is true, by design the probability of a type I error is $\alpha$
    \item \textbf{Type II error:} Not rejecting $H_0$ when it is false, the probability of a type II error is $\beta$ based on the true distribution of the test statistic
\end{itemize}
The errors are illustrated in figure \ref{fig:type_1_2_error}.

\begin{figure}
    \centering
    \includesvg[width=0.95\textwidth]{figures/table_hypothesis.svg}
    \caption{Truth table of hypothesis testing.}
    \label{fig:type_1_2_error}
\end{figure}

The most powerful test - the one with the highest probability of correctly rejecting $H_0$ at a given $\alpha$ - is chosen based on the Neyman-Pearson lemma\footnote{\textit{Roughly it says that a likelihood ratio test is the way to go for testing simple hypotheses against each other.}}.

\subsubsubsection{Bayesian perspective on hypothesis testing}
In Bayesian statistic we have the full posterior of the parameter of interest, so we can
directly make statements about the parameters.

\subsubsection{Overview on specific test procedures}
\begin{itemize}
    \item \textcolor{blue1}{Exact tests}: The underlying probability distribution is exactly known under $H_0$ (e.g. the binomial distribution for a coin flip). No further parameters need to be estimated $\rightarrow$ \textit{non-parametric}.
    \item \textcolor{blue1}{Asymptotic tests}: Based on the central limit theorem, the distribution of the test statistic is derived under $H_0$. Parameters are estimated from the data as necessary.
    \item \textcolor{blue1}{Bootstrapping}: The distribution of the test statistic is derived empirically (empirical distribution function) by resampling the data.
\end{itemize}

\subsubsection{Hypothesis and statistics for typical tests}
Consider table \ref{tab:typical_tests}.

\begin{table}
    \centering
    \begin{tabular}{|p{0.2\linewidth}|p{0.15\linewidth}|p{0.15\linewidth}|p{0.15\linewidth}|p{0.2\linewidth}|}
        \hline 
        Test For & Null Hypothesis $(H_0)$ & Test Statistic & Distribution & Use When \\
        \hline 
        Population mean $(\mu)$ if variance $\sigma^2$ is known & $$\mu=\mu_0$$ & $$\frac{\left(\bar{x}-\mu_o\right)}{\sigma / \sqrt{n}}$$ & $\mathcal{N}$ & Normal distribution or $n>30 ; \sigma$ known \\
        \hline 
        Population mean $(\mu)$ if variance $\sigma^2$ must be estimated & $$\mu=\mu_0$$ & $$\frac{\left(\bar{x}-\mu_o\right)}{s / \sqrt{n}}$$ & $t_{n-1}$ & $n<30$, and/or $\sigma$ unknown \\
        \hline
        Difference of two means $(\mu_1-\mu_2)$ with $\sigma_1^2,\sigma_2^2$ known & $$\mu_1-\mu_2=0$$ (Test for $=c$ by replacing the $0$ accordingly.) & $$\frac{\left(\bar{x}_1-\bar{x}_2\right)-0}{\sqrt{\frac{\sigma_1^2}{n_1}+\frac{\sigma_2^2}{n_2}}}$$ & $\mathcal{N}$ & Both normal distributions, or $n_1, n_2 \geq 30 ; \sigma_1, \sigma_2$ known \\
        \hline 
        Difference of two means $(\mu_1-\mu_2)$ with $\sigma_1^2,\sigma_2^2$ estimated& $$\mu_1-\mu_2=0$$ & $$\frac{\left(\bar{x}_1-\bar{x}_2\right)-0}{\sqrt{\frac{s_1^2}{n_1}+\frac{s_2^2}{n_2}}}$$ & $t$ distribution with $df=$ the smaller of $n_1-1$ and $n_2-1$ & $n_1, n_2<30 ;$ and/or $\sigma_1, \sigma_2$ unknown \\
        \hline 
        Mean difference $\mu_d$ (of paired data $\{x^{(1)}_i,x^{(2)}_i\}_{i=1}^N, d_i=x^{(1)}_i-x^{(2)}_i$) & $$\mu_d=0$$ & $$\frac{\left(\bar{d}-\mu_d\right)}{s_d / \sqrt{n}}$$ & $t_{n-1}$ & $n<30$ pairs of data and/or $\sigma_d$ unknown \\
        \hline
    \end{tabular}
    \caption{Hypothesis and test statistic for typical tests}
    \label{tab:typical_tests}
\end{table}

\subsubsection{Simple example for an application case of hypothesis testing}

Consider survival times of patients with breast cancer (sample $1$) and stomach cancer (sample $2$).

\begin{equation}
    \begin{aligned}
        D^{(1)} &= (107, 353, 1764, 667, 990, 78, 667, 44, 9, 27), \quad n_1 = 10 \\
        D^{(2)} &= (374, 253, 812, 246, 95, 367, 251, 309, 594, 826, 593, 97), \quad n_2 = 12
    \end{aligned}
\end{equation}

We will work with the logged lifetimes $X^{(1)}, X^{(2)}, x^{(k)}_j=\log{d^{(k)}_j}$ (see e.g. \cite{over04}) and present kernel density estimates of
both the original and log-transformed data in figure \ref{fig:surv_kde}. Note that in the sample $1$ we have an outlier at $1764$ which strongly
pulls the mean of the non-logged data to higher values. The log-transforms crunches down especially the outlier, shifting the mean to lower values.

\begin{figure}[!htb]
    \centering
    \includesvg[width=1\linewidth]{figures/kde_log_kde.svg}
    \caption{Kernel Density Estimates of the original and log-transformed data.}
    \label{fig:surv_kde}
\end{figure}

\textbf{Question:} Do patients with stomach cancer live significantly longer than patients with breast cancer?

As $\bar{X}^{(1)}$ is smaller than $\bar{X}^{(2)}$ (see figure \ref{fig:surv_kde}), we will consider $\mu_2 - \mu_1$.

\textbf{Null hypothesis:} Therefore our \textit{no-difference hypothesis is:} $H_0:\mu_2 - \mu_1 \leq c$, some constant $c$.

\textbf{Intuition}: While $\bar{X}^{(2)} - \bar{X}^{(1)} > 0.25$ we cannot reject $H_0$ fundamentally 
as of the large stray in the distributions (see figure \ref{fig:surv_kde} again).

\textbf{Test statistic:} The basic reasoning here is that we want an under $H_0$ centered test distribution,
which takes into consideration that the scale on which means are to be compared is given
by the standard error of the means (see again figure \ref{fig:surv_kde}, if the distributions
would be more peaked, maybe $H_0$ could be rejected).

\begin{equation}
    \begin{multlined}
    t_{\text{obs}} = \frac{\bar{X}^{(2)}_{\text{sample}} - \bar{X}^{(1)}_{\text{sample}} - c}{S_p\sqrt{\frac{1}{n^{(1)}} + \frac{1}{n^{(2)}}}} \\
    S_p = \sqrt{\frac{\left(n^{(1)} - 1\right)\left(S^{(1)}\right)^2 + \left(n^{(2)} - 1\right)\left(S^{(2)}\right)^2}{n^{(1)} + n^{(2)} - 2}} \\
    \end{multlined}
\end{equation}

The next step would be to calculate (derive / bootstrap) the distribution of the test statistic and see
where the observed value falls. But before we get to this, we consider 
\textit{exact tests}.

\subsubsection{Exact tests}
Here based on $H_0$ the distribution of the test statistic is exactly known.

\subsubsubsection{Sign test - testing the direction of change}
Consider \textbf{paired} observations $\{ x_i \in X, y_i \in Y \}_{i=1}^N$, e.g. Covid19 infection rates before
($x_i$) and after ($y_i$) some restriction measures across multiple countries $i=1,\ldots,N$.
\paragraph*{Sign testing procedure}
\begin{enumerate}
    \item We want to test if observations in $X$ are larger than those in $Y$ (e.g. Covid-measures
    decreased infection rates) (or vice versa), ties are ignored / disregarded.
    \item Consider the statistic $T_i = \sign(x_i-y_i)$ only based on the sign of the difference.
    With this
    \begin{equation}
        H_0: E[T] = 0, \quad H_1: E[T] > 0
    \end{equation}
    or equivalently $H_0: p(T = +1) = 0.5, \quad H_1: p(T = +1) > 0.5$. Consider the test statistic
    \begin{equation}
        k_0 = \frac{1}{2} \sum_{i=1}^N \left( T_i + 1 \right)
    \end{equation}
    counting the number of positive signs.
    \item Under $H_0$ we follow $k_0 \sim \text{Binomial}(N,0.5)$
    \item We choose a significance level $\alpha = 0.05$
    \item We want to reject $H_0$ on the right side so towards $H_1: E[T] > 0$. Therefore
    the $p$-value of observing a $k$ more extreme than $k_0$ is (under $H_0$)
    \begin{equation}
        p\left(k \geq k_0 \mid H_0\right)=\sum_{i=k_0}^N\left(\begin{array}{c}
        N \\
        i
        \end{array}\right)\left(\frac{1}{2}\right)^k\left(\frac{1}{2}\right)^{N-k}=\sum_{i=k_0}^N\left(\begin{array}{c}
        N \\
        i
        \end{array}\right)\left(\frac{1}{2}\right)^N
        \end{equation}
    \item We accept $H_1$ if $p \leq \alpha$
    \note{In the one-sided scenario the \enquote{we neglect the null hypothesis and therefore accept the alternative}
    is not a nice formulation, if $H_0$ and $H_1$ are not exhaustive.}
\end{enumerate}

\redbox{\textbf{Problem of confounding factors:} Note that we strictly only make statements about data.
E.g. the viral load decreased between measurements. If we want to make a statement like giving the
drug caused the decrease, we need to control for confounding factors. For instance the reason for the change
in viral load could also be that one measurement was in the morning, the other in the evening.}

\paragraph*{Pro and con - sign test}
\begin{itemize}
    \item \textcolor{green1}{Pro}: as in general in exact tests, no further assumptions about the distribution of the statistic are done, except that $H_0$ holds
    \item \textcolor{red1}{Con}:
    \begin{itemize}
        \item only sign information is used, not how large the differences are, not powerful, high $\beta$
        \item other relations like correlations between $x_i$ and $y_i$ are not considered
    \end{itemize}
\end{itemize}

\subsubsubsection{Mann-Whitney U test (aka Wilcoxon rank-sum test) - no strict pairing necessary}
\textbf{Aim:} Test if two independent samples come from the same distribution. It is an 
alternative to the \textit{unpaired two-sample t-test}.
\textbf{Example Question:} Do marks of boys and girls in a test differ?
\textbf{Concept:} The test is done base on ranks-sums. Order all
observations from both samples in ascending order and assign ranks to them\footnote{For ties both get the same rank halfway between the next lower and higher rank.}. 
Then sum the ranks of the observations from each sample respectively. One can find an exact distribution of the test statistic under $H_0$.

\subsubsection{Asymptotic test}
\subsubsubsection{Central limit theorem}
Let $X_i, i=1,\ldots,N$ be independent random variables with 
variance $\sigma^2$ and finite mean $\mu := E[X_i] < \infty$\footnote{Consider
I am in front of a line and kick a ball into a random angle $\in (-90°, 90°)$.
Neither the mean nor the standard deviation of the position of the shots along the line
will ever converge. If I am $1$ away from the line, this is described by the standard
Cauchy distribution $f(x) = \frac{1}{\pi \cdot (1+x^2)}$.}. Then
\begin{equation}
    \lim _{N \rightarrow \infty} \frac{\frac{1}{N} \sum_i\left(X_i-\mu\right)}{\sigma / \sqrt{N}} \sim \mathcal{N}(0,1)
\end{equation}
so when we repeatedly sample from some distribution (e.g. an exponential one)
and calculate the mean of those samples, then those means will be normally distributed,
as e.g. shown in figure \ref{fig:clt_exponential}.

\begin{figure}[!htb]
    \centering
    \includesvg[width=0.7\textwidth]{figures/clt_exponential.svg}
    \caption{Central limit theorem for the exponential distribution.}
    \label{fig:clt_exponential}
\end{figure}

\note{If $\mu$ and $\sigma$ are unknown, this - in the large $N$ limit - still holds true when the estimates of $\mu$ and $\sigma$ are used,
even though the estimate of $\sigma$ is (potentially) biased, by \textit{Slutskys theorem} (for which it's enough that the estimate of 
$\sigma$ converges in probability to $\sigma$). For smaller $N$ we have to be more careful.}

\greenbox{\textbf{Usage in testing:} Being able to estimate the sampling distribution of a mean just from a sample is very useful when
we want to compare means.}

In the following we will introduce some important asymptotic distributions and asymptotic tests.

\subsubsubsection{Important distributions in testing I: $\chi^2$ distribution}
The sum of squares of a normally distributed variable $Z_i \sim \mathcal{N}(0,1)$ ($\{ Z_i \}_{i=1}^N$ i.i.d.) follows
\begin{equation}
    \sum_{i=1}^K Z_i^2 \sim \chi^2(k)
\end{equation}
where $N$ is the number of degrees of freedom (in the sample not the parameters),
here the number of independent random variables. The $\chi^2$ distribution is shown in figure \ref{fig:chi2}.

\begin{figure}[!htb]
    \centering
    \includesvg[width=0.7\textwidth]{figures/chi2.svg}
    \caption{The $\chi^2$ distribution for different degrees of freedom (numbers drawn per sample). For large $N$ we should approach a normal distribution again.}
    \label{fig:chi2}
\end{figure}

\subsubsubsection{Important distributions in testing II: $t$-distribution}
The $t_N$ distribution is defined on the ratio
\begin{equation}
    t_N = \frac{z}{\sqrt{\chi_N^2 / N}}, \quad z \sim \mathcal{N}(0,1)
\end{equation}
where $z$ and $\chi_N^2$ are independent. The $t$-distribution is shown in figure \ref{fig:t_dist}.

\begin{figure}[!htb]
    \centering
    \includesvg[width=0.7\textwidth]{figures/t.svg}
    \caption{The $t$-distribution for different degrees of freedom (numbers drawn per sample). For large $N$ we should approach a normal distribution again.}
    \label{fig:t_dist}
\end{figure}

\subsubsubsection{The Chi-Square Test - test on counts}
\paragraph*{Setting} Consider an experiment of counts of categories, i.e. when tossing
balls at boxes assuming non-changing probabilities of the categories (boxes) and
independent throw.

\paragraph*{Hypothesis} We make a hypothesis for all the probabilities of the 
categories $p_i, i=1,\ldots,k$. For $n$ total throws, we expect (binomial distribution)
$E(n_i) = n p_i$ hits per box in $n$ throws (counts per category).

\paragraph*{Test statistic} The test statistic is the sum of squares of the differences
between the observed and expected counts, normalized by the expected counts.

\begin{equation}
    X^2=\sum_{i=1}^k \frac{\left[n_i-E\left(n_i\right)\right]^2}{E\left(n_i\right)}=\sum_{i=1}^k \frac{\left[n_i-n p_i\right]^2}{n p_i} \underset{\text{N large}}{\sim} \chi^2_{df} \underset{\text{N large, approx.}}{\sim} \chi_{k-1}^2
\end{equation}

\yellowbox{\textbf{Why does this follow a $\chi^2$ distribution with $k-1$ degrees of freedom?}: For large $n$, $n_i$ will be approximately normally distributed, while $n$ and $p_i$ are just constants.
While we sum over $k$ categories, we have the constraint that the $n_i$ sum up to $n$, so one degree of freedom is lost (equivalently $\sum p_i = 1$).}

Given e.g. tabulated data of the $\chi^2$ distribution we can, based on the observed $X^2$ value, calculate the $p$-value and decide if we reject $H_0$.

\subsubsubsection{Students $t$-test}
We finally come back to tests on means, as introduced for the example of the survival times of breast and stomach cancer patients 
(a \textit{two-sample t-test}).
\bluebox{\textbf{Aim in t-tests:}
\begin{itemize}
    \item 1-sample t-test: Test a sample might come from a distribution with mean $\mu_0$
    \item 2-sample t-test: Test whether two distributions come from a distribution with the same mean, e.g. $H_0:$ breast and 
    stomach cancer patients have the same survival times.
\end{itemize}
}
\note{Formulations where we test on a certain difference between means follow analogously.}

\paragraph*{One-sample t-test} Consider we have measured $\{ x_i \}, i=1,\ldots,N$ and want to test against a known means $\mu_0$.
\begin{equation}
    H_0: \mu = \mu_0, \quad H_1: \mu \neq \mu_0, \quad \bar{x} = \frac{1}{N} \sum_{i=1}^N x_i
\end{equation}
We use the test statistic
\begin{equation}
    \begin{gathered}
        t = \frac{\bar{x} - \mu_0}{\hat{\sigma}_{\bar{x}}} \sim t_{N-1}, \quad \text{std. of mean } \hat{\sigma}_{\bar{x}} = \frac{s}{\sqrt{N}} \\
        \text{corrected (but biased) sample std. } s = \sqrt{\frac{1}{N-1} \sum_{i=1}^N (x_i - \bar{x})^2}
    \end{gathered}
\end{equation}
\note{While for large $N$ we have a normal distribution by Slutkys theorem, for lower $N$ we have to mind that $\hat{\sigma}_{\bar{x}}$ is an estimate.}
The distribution according to $t_{N-1}$ can be seen from
\begin{equation}
    t=\frac{\bar{x}-\mu_0}{\hat{\sigma}_{\bar{x}}}=\frac{\left(\bar{x}-\mu_0\right) /(\sigma / \sqrt{N})}{\hat{\sigma}_{\bar{x}} /(\sigma / \sqrt{N})}=\frac{\overbrace{\left(\bar{x}-\mu_0\right) /(\sigma / \sqrt{N})}^{\sim \mathcal{N}(0,1)}}{\underbrace{\sqrt{\frac{1}{N-1} \sum_{i=1}^N\left(\frac{x_i-\bar{x}}{\sigma \slash \sqrt{N}}\right)^2}}_{\sim \sqrt{\chi_{N-1}^2 \slash (N-1)}}} \sim t_{N-1}
\end{equation}
\greenbox{\textbf{Why is it $N-1$?} The reasoning for why it is $t_{N-1}$ here is the same as for
the reasoning of $N-1$ in the Bessel correction. While we have $N$ independent observations in the sample
we only have $N-1$ independent residuals, $x_i - \bar{x}$ as they must sum up to $0$ (by definition of the sample mean).}

\paragraph*{Two-sample t-test} Consider we have two independent samples $X_1$ and $X_2$ of (possibly different) sizes $N_1$ and $N_2$ with means
\begin{equation}
    \bar{x}_1 = \frac{1}{N_1} \sum_{i=1}^{N_1} x_{1_i}, \quad \bar{x}_2 = \frac{1}{N_2} \sum_{i=1}^{N_2} x_{2_i}
\end{equation}
We want to test if the means \textbf{of the populations they come from} $\mu_1, \mu_2$ are significantly different (or e.g. larger than a threshold).
\begin{equation}
    H_0: \mu_1 = \mu_2, \quad H_1: \mu_1 \neq \mu_2
\end{equation}

For unequal sample sizes and similar variances, we get
\begin{equation}
    \begin{gathered}
        t=\frac{\left(\bar{x}_1-\mu_1\right)-\left(\bar{x}_2-\mu_2\right)}{\hat{\sigma}_{\bar{x}_1-\bar{x}_2}} \underset{H_0: \mu_1 = \mu_2}{\approxeq} \frac{\bar{x}_1-\bar{x}_2}{\hat{\sigma}_{\text {pool }} \sqrt{\frac{1}{N_1}+\frac{1}{N_2}}} \sim t_{N_1+N_2-2}
    \end{gathered}
\end{equation}
where the pooled standard deviation $\hat{\sigma}_{\text{pool}}$ is given by
\begin{equation}
    \hat{\sigma}_{\text {pool }}=\sqrt{\frac{\left(N_1-1\right) \hat{\sigma}_1^2+\left(N_2-1\right) \hat{\sigma}_2^2}{N_1+N_2-2}},\quad \left(\begin{array}{c}
    \text { makes sense for } \\
    \frac{1}{2}<\frac{\hat{\sigma}_1^2}{\hat{\sigma}_2^2}<2
    \end{array}\right)
\end{equation}
where $\hat{\sigma}_1^2, \hat{\sigma}_2^2$ are the unbiased estimates of the population variances.

\paragraph*{Why this complicated denominator and t-distribution?} The idea is that means have to be
compared on the scale of their standard errors, as illustrated in figure \ref{fig:means_diff_std}.

\begin{figure}[!htb]
    \centering
    \includesvg[width=0.7\textwidth]{figures/means_diff_std.svg}
    \caption{It is not the absolute difference between means that counts.}
    \label{fig:means_diff_std}
\end{figure}

\subsubsubsection{Important distributions in testing III: $F$-distribution}
Let $\chi_1^2 \sim \chi_{N_1}^2$ and $\chi_2^2 \sim \chi_{N_2}^2$ be independent. Then the ratio
\begin{equation}
    F = \frac{\chi_1^2 \slash N_1}{\chi_2^2 \slash N_2} \sim F_{N_1,N_2}
\end{equation}
of these two $\chi^2$ distributed random variables divided by their degrees of freedom follows the $F$-distribution.

\subsubsubsection{The $F$-test - test on variances}
Assuming two independent samples $X_1$ and $X_2$ from two populations 
(e.g. test scores of boys and girls), which we assume to come from populations
with normal distribution with means $\mu_1, \mu_2$ and variances $\sigma_1^2, \sigma_2^2$.

Based on the samples
\begin{equation}
    \{ x_{1_i} \}_{i=1}^{N_1}, \quad \{ x_{2_i} \}_{i=1}^{N_2}
\end{equation}
we want to test if the population variances 
can reasonably be assumed to be equal, i.e. $H_0: \sigma_1^2 = \sigma_2^2$.

The test statistic is given by

\begin{equation}
    F = \frac{s_1^2}{s_2^2} \sim F_{N_1-1,N_2-1}, \quad s_1^2 = \frac{1}{N_1-1} \sum_{i=1}^{N_1} (x_{1_i} - \bar{x}_1)^2, \quad s_2^2 = \frac{1}{N_2-1} \sum_{i=1}^{N_2} (x_{2_i} - \bar{x}_2)^2
\end{equation}

\paragraph*{Example of comparing two models based on explained variance} Consider data
\begin{equation}
    D=\left\{y_i, \vec{x}_i\right\}, \quad \vec{x}_i=\left(\begin{array}{c}
    x_{i 1} \\
    \vdots \\
    x_{i p}
    \end{array}\right)
\end{equation}
on which we try a full and reduced model
\begin{equation}
    \begin{gathered}
        \text { full model }(I): y_i=\beta_0+\sum_{j=1}^p \beta_j^{(I)} x_{i j}+\epsilon_i \\
        \text { reduced model (II): } y_i=\beta_0+\sum_{j=1}^q \beta_j^{(I I)} x_{i j}+\epsilon_i, \quad q<p
    \end{gathered}
\end{equation}
and make the null hypothesis
\begin{equation}
    H_0: \beta_{q+1} = \dots = \beta_p = 0
\end{equation}
i.e. that the small model is sufficient for linearly explaining the data.

Now consider the model losses
\begin{equation}
    \operatorname{SSQ}(\vec{\beta}^{(I)})=\sum_{i=1}^N\left(y_i-\hat{y}_i^{(I)}\right)^2, \quad \operatorname{SSQ}(\vec{\beta}^{(I I)})=\sum_{i=1}^N\left(y_i-\hat{y}_i^{(I I)}\right)^2
\end{equation}

\note{By choice of the residuals, $y_i - \hat{y}_i^{(I,II)} \sim \mathcal{N}(0,\sigma^2)$, so the sums of squares of the residuals
$\operatorname{SSQ}(\vec{\beta}^{(I,II)})$ follow $\chi^2$ distributions.}

Therefore the test statistic comparing the sum of squares of the residuals of the two models is given by
\begin{equation}
    \begin{gathered}
        \mathcal{F} = \frac{\left(\operatorname{SSQ}(\vec{\beta}^{(I I)}) - \operatorname{SSQ}(\vec{\beta}^{(I)}) \right)\slash (p-q)}{\operatorname{SSQ}(\vec{\beta}^{(I)}) \slash (N-p-1)} \sim F_{p-q,N-p-1} \\
        \operatorname{SSQ}(I I) - \operatorname{SSQ}(I) \geq 0 \text { as the smaller model has the higher loss }
    \end{gathered}
\end{equation}

\bluebox{\textbf{Why do we divide the difference in the SSQs by $p-q$?} Note that we always divide by the degrees of freedom of the term
of consideration. Consider $p=q$ - we then have the same models, so the SSQ difference has no degree of freedom but must be zero. The
more different the models, the more degrees of freedom the difference has.}

\bluebox{\textbf{Why do we divide the SSQ of the full model by $N-p-1$?} The sum of the residuals has less degrees of freedom as of the parameters
estimated and $-1$ as the residuals must sum up to zero.}

\subsubsubsection{Likelihood ratio test principle - comparing models by likelihood}
\bluebox{\textbf{Aim of the likelihood ratio test:} We want to compare a full to a restricted model,
to see, if the full more complex model should be used or if the simpler one is just fine.}

So consider a model $M_\vec{\theta}$ with parameters $\vec{\theta} \in \Omega$ (the parameter space).
The resctricted model $M_{\vec{\theta}_0}$ has parameters from a limited parameter space $\Omega_0 \subset \Omega$,
constrained by the null hypothesis $H_0$.

\paragraph*{Test statistic} We compare the likelihood of the sample under the full and restricted model.
\begin{equation}
    \lambda = \frac{\mathcal{L}_X(\vec{\hat{\theta}}_{\text{MLE}} \in \Omega_0)}{\mathcal{L}_X(\vec{\hat{\theta}}_{\text{MLE}} \in \Omega)} \in [0,1]
\end{equation}
If the null hypothesis is true, then the limitation of $\Omega$ has no effect on the likelihood
\begin{equation}
    \text{if } H_0: \vec{\hat{\theta}}_{\text{MLE}} \in \Omega_0 \text{ in general } \quad \rightarrow \quad \lambda = 1
\end{equation}
As we usually want a test statistic distributed around $0$, we use the log-likelihood ratio
\begin{equation}
    \begin{gathered}
        D := -2 \log{\lambda} \underset{\text{for sample size } \geq 10 \text{ as of CLT}}{\sim} \chi^2_{k-k_0} \\
        \quad k = \text{number of parameters in the full model}, \quad k_0 = \text{number of parameters in the restricted model} \\
        df = k - k_0 \text{ \# parameters fixed by the null hypothesis}
    \end{gathered}
\end{equation}
\textbf{Example:} For instance we could have the baseline linear model
\begin{equation}
    M_\Omega: y_i = \beta_0 + \beta_1 x_i + \beta_2 x_i^2 + \epsilon_i, \quad \beta_1, \beta_2 \in \Omega
\end{equation}
and the null hypothesis $H_0: \beta_2 = 0$. Then
\begin{equation}
    M_{\Omega_0}: y_i = \beta_0 + \beta_1 x_i + \epsilon_i \quad \rightarrow \quad df = 1
\end{equation}

\note{To really approximately have a $\chi^2$ distribution, it might make sense to bring the baseline 
observation to a more Gaussian form, e.g. by Box-Cox transformation, generally
\begin{equation}
    z_i = \begin{cases}\frac{x_i^\lambda-1}{\lambda} & \text { if } \lambda \neq 0 \\ \ln y_i & \text { if } \lambda=0\end{cases}
\end{equation}
with $\lambda$ being a parameter. Where for $\lambda = 0$ we have the log-transformation which can sometimes
help to unskew distributions (stabilize the variance) (it it was log-normal previously) (might also have the opposite effect).
}

\subsection{Bootstrap methods}
\bluebox{Uses of Bootstrap include: 
\begin{itemize}
    \item estimating the standard error of a statistic
    \item estimating the bias of a statistic
    \item estimating confidence intervals for a statistic
    \item estimating the distribution of a statistic for hypothesis testing
\end{itemize}
}

At the heart of bootstrap lies random sampling with replacement to better assess statistical
estimates (assign measures of accuracy\footnote{For the mean e.g. a measure of accuracy is the standard error for which we have the explicit formula $\SE = \sqrt{\frac{s^2}{n}}$, $\bar{X}_{\text{sample}} = \frac{1}{n} \sum_{i=1}^{n} x_i$, $s^2 = \frac{1}{n-1} \sum_{i=1}^{n} (x_i - \bar{X}_{\text{sample}})^2$, $X_{\text{sample}} = (x_1,\dots,x_n)$. For most other statistics (e.g. the median) we do not have such an explicit formula. See \cite[chapter 1]{Efron1994}.}) - we use a sample as an estimate of a population and bootstrap samples from the sample
as estimates of the sampling distribution. Consider we have observed a sample $X_{\text{sample}} = (x_1,\dots,x_n)$ and are interested in a population parameter
$\theta$ which we can estimate as $\hat{\theta}$ from a sample. The non-parametric bootstrap method is given by
\begin{enumerate}
    \item $\forall b \in \{1, \dots, B\}$ (where $B$ is the number of bootstrap samples): Generate a bootstrap sample $X^{(b)} = (x_1^{(b)},\dots,x_n^{(b)})$ by sampling from $X$ with replacement and calculate $\hat{\theta}^{(b)}$ on the bootstrap sample.
    \item This yields an empirical distribution of estimates $\hat{\theta}^{(1)},\dots,\hat{\theta}^{(B)}$ from which we can calculate the mean $\hat{\theta}^{*}$ and standard error $\SE\left(\hat{\theta}^*\right)$.
\end{enumerate}
with the bootstrap estimate and the estimated standard error given by
\begin{equation}
    \begin{aligned}
        \hat{\theta}^{*} &= \frac{1}{B} \sum_{b=1}^{B} \hat{\theta}^{(b)} \\
        \hat{\SE}(\hat{\theta}) = \SE\left(\hat{\theta}^*\right) &= \sqrt{\frac{1}{B-1} \sum_{b=1}^{B} \left( \hat{\theta}^{(b)} - \hat{\theta}^{*} \right)^2}
    \end{aligned}
\end{equation}

The bootstrap estimate of the $\bias\left(\hat{\theta}\right) = E\left[ \hat{\theta} \right] - \theta$ of the estimator $\hat{\theta}$ is given by
\begin{equation}
    \hat{\bias}\left(\hat{\theta}\right) = \hat{\theta}^{*} - \hat{\theta}
\end{equation}

\note{By the bias correction we introduce a new source of variance (the variance of the bias) which we do not account for in the standard error.
Using further bootstrapping we can estimate the standard error of $\bar{t}$, but if $\hat{\text{bias}}(\hat{t})$ is small compared to $\hat{\SE}\left(\hat{t}\right)$
it is safer to use $\hat{t}$ than $\bar{t}$ (see \cite[chapter 10.6, p. 138]{Efron1994}).}

\subsubsection{Caveats of Bootstrap\skipthis}
Beware, that bootstrap is not magic - it will not retrieve information not present in
a sample or fix a bad sample. Also, the bootstrap sample is generally centered at
the observed statistic, not the population parameter (e.g. $\bar{X}$ not $\mu$) - using
bootstrap we will not improve on $\bar{X}$ \citep[section 2.3 (also see there for exceptions)]{tim15}. And if there is bias
(as estimated above), the bootstrap estimate is indeed not a bias-corrected estimate - if $\hat{\theta}^{*}$
is greater than $\hat{\theta}$, the bias corrected estimate $\bar{\theta} = \hat{\theta} - \hat{\bias}$ should be less than $\hat{\theta}$,
and this kind of bias correction also has pitfalls\footnote{$\bar{\theta}$ might have a much larger standard error than $\hat{\theta}$.} \citep[chapter 10.6, p. 138]{Efron1994}. Simply put, if from resampling a sample we get a positive bias
then to get to the unbiased one we have to go "two steps back", $\hat{\theta} = \hat{\theta}^{*} - 2 \cdot (\hat{\theta}^{*} - \hat{\theta}) = 2\hat{\theta}-\hat{\theta}^{*}$ (eq. 10.41 in \cite{Efron1994}).

In figure \ref{fig:bootstrap_samples} we illustrate that as the sample carries distributional information so do the bootstrap samples. In figure \ref{fig:boots_corr} we illustrate the difference between the sample
estimate, bootstrap estimate and corrected estimate for a biased estimator, the standard deviation. Here the bias comes from the fact that in the calculation
of the standard deviation we also estimate the mean from which the points in the sample naturally deviate less than from the true mean, so the
standard deviation is (with the biased estimator) underestimated.

\yellowbox{\textbf{Note on the necessary sample size:} Note also that in non-parametric bootstrapping, we 
extract the distributional form empirically from the sample (different from the exact test
where the distribution is known by $H_0$ and asymptotic tests where the distribution
is assumed with necessary parameters estimated from the data) -
bootstrapping often requires a larger sample size.}

\begin{figure}[!htb]
    \centering
    \includesvg[width=.95\linewidth]{figures/boot_samples.svg}
    \caption{Bootstrap Samples from a sample.}
    \label{fig:bootstrap_samples}
\end{figure}

\begin{figure}[!htb]
    \centering
    \includesvg[width=.95\linewidth]{figures/boot_cav.svg}
    \caption{Illustration of the difference between the sample estimate, bootstrap estimate and corrected estimate for the biased estimate of the standard deviation.}
    \label{fig:boots_corr}
\end{figure}

\subsubsection{Bootstrap Confidence Intervals\skipthis}
Note in general that bootstrap distributions tend to be narrow on average, so the bootstrap confidence intervals often under-cover \citep{tim15}.,
as values from the tails of the distribution underlying distribution are less likely to be in our sample from 
which we draw the bootstrap samples in the first place.

\note{This also carries over to testing. When a bootstrap distribution is used as an alternative
to a $t$-distribution it is usually more narrow, so the bootstrap test rejects $H_0$ more often than the $t$-test.}

\subsubsection{Standard Normal Bootstrap Confidence Interval}
If $\hat{\theta}$ is a sample mean, then for large sample sizes, we can apply the central limit theorem

\begin{equation}
    \frac{\hat{\theta} - \theta}{\SE(\hat{\theta})} \isEquivTo{a} N(0,1)
\end{equation}

so based on the bootstrap estimate of the standard error $\hat{\SE}(\hat{\theta})$ we can calculate a $1-\alpha$-confidence interval for $\theta$ as

\begin{equation}
    \hat{\theta} \pm z_{1 - \alpha/2} \hat{\SE}(\hat{\theta}), \quad z_{1 - \alpha/2} \text{ is the } 1 - \alpha/2 \text{ quantile of } N(0,1)
\end{equation}

assuming that $\hat{\theta}$ is approximately normally distributed, $\hat{\theta}$ is unbiased and $\hat{\SE}(\hat{\theta})$ is a good estimate of $\SE(\hat{\theta})$.

\subsubsubsection{Percentile Bootstrap Confidence Interval\skipthis}
Here, we use quantiles based on the empirical distribution of the bootstrap samples, so a $1-\alpha$-confidence interval for $\theta$ is given by
\begin{equation}
    \begin{multlined}
        \left[ \hat{\theta}^*_{\alpha/2}, \hat{\theta}^*_{1 - \alpha/2} \right], \quad \hat{\theta}^*_{\alpha/2} \text{ is the } \alpha/2 \text{ quantile of } \hat{\theta}^*, \\
        \text{ with } \hat{\theta}^*_{1 - \alpha/2}, \hat{\theta}^*_{1 - \alpha/2} \text{ being the quantiles of the empirical distribution } \left\{ \hat{\theta}^{(b)} \right\}_{b=1,\dots,B}
    \end{multlined}
\end{equation}

% \subsubsubsection{Basic Bootstrap Confidence Interval\skipthis}
% Akin to the bias-corrected estimate $\bar{\theta}$, we get a basic bootstrap confidence interval for $\theta$ as
% \begin{equation}
%     \begin{multlined}
%         \left[ 2\hat{\theta} - \hat{\theta}^*_{1 - \alpha/2}, 2\hat{\theta} - \hat{\theta}^*_{\alpha/2} \right], \quad \hat{\theta}^*_{\alpha/2} \text{ is the } \alpha/2 \text{ quantile of } \hat{\theta}^*, \\
%         \text{ with } \hat{\theta}^*_{1 - \alpha/2}, \hat{\theta}^*_{1 - \alpha/2} \text{ being the quantiles of the empirical distribution } \left\{ \hat{\theta}^{(b)} \right\}_{b=1,\dots,B}
%     \end{multlined}
% \end{equation}

\subsubsection{Bootstrap Hypothesis Testing}
\label{subsec:bootstrap_hypothesis_testing}

Consider we want to test a null hypothesis $H_0$ and we have a test statistic $T$ where observed values of $T$ under $H_0$ can be calculated for
a sample as $t_{\text{obs}}\left( X_{\text{sample}} \right)$. In a setting where we make assumptions on the distributions
of the test statistic we would test based on where our observed value falls in the distribution of the test statistic under $H_0$.

\problem{What if the sampling distribution of the test statistic is not known? What if assumptions of asymptotic tests are not met?}

\bluebox{\textbf{Intermezzo - general empirical distribution function:} Consider we have observed data $X = (x_1,\dots,x_n)$ from some underlying
population distribution $F$. The distribution information of $F$ lies in how often similar values of $x_i$ occur in the sample $X$. We therefore
estimate the cumulative distribution function $F$ by the empirical distribution function $\hat{F}_n$ as
\begin{equation}
    \hat{F}_n(x) = EDX(x) = \frac{\#\{ x_i \leq x \}}{N}
\end{equation}
which resembles the true distrution in the large $N$ limit.}

\idea{While we only have one sample at hand, by resampling we can generate multiple \textit{samples} and based on those samples
\textbf{conditioned on $H_0$} find an empirical distribution of the test statistic under $H_0$. See figure \ref{fig:bootstrap}.}

\note{The re-samples / empirical distribution must be constructed / adapted to follow $H_0$. So we want to use the distributional
information of the sample to generate new samples that are consistent with $H_0$. If we would just directly calculate 
our test statistic on resampled samples and then our observed test statistic by the same formula - why should we expect
to ever reject the null hypothesis?}

In table \ref{tab:test_overview}, the different kinds of tests with respect how the test-statistic distribution
is constructed are compared.

\begin{table}[!htb]
    \centering
    \begin{tabular}{|p{0.2\linewidth}|p{0.2\linewidth}|p{0.2\linewidth}|p{0.2\linewidth}|}
        \hline
        & \textbf{Exact test} & \textbf{Parametric test} & \textbf{Nonparametric bootstrap test} \\
        \hline
        \textcolor{blue1}{Knowledge on the distribution function $F$ of the test statistic} & $F_\alpha(T)$ (/ the parameters $\alpha$) is known 
        exactly based on $H_0$ & In $F_{\hat{\alpha}}(T)$ some parameters have to be estimated, e.g. the STD $\hat{\sigma}$ & The distribution $\hat{F}$ itself is estimated using the data
        constrained to $H_0$ \\
        \hline
    \end{tabular}
    \caption{Comparison of different kinds of tests.}
    \label{tab:test_overview}
\end{table}

Note that there is also
\begin{itemize}
    \item \textbf{Parametric bootstrap test:} The sample is assumed to come from a known distribution with unknown parameters. These parameters are estimated from the sample and then the distribution is used to generate bootstrap samples and to construct the \textit{empirical distribution of the test statistic under $H_0$}.
    \item \textbf{Semiparametric bootstrap test:} E.g. do the usual bootstrap resampling but add noise (e.g. $\sim \mathcal{N}(\sigma^2)$)
\end{itemize}

\subsubsubsection{1-sample test as an introductory example}
Consider for instance for a sample $X_{\text{sample}} = (x_1,\dots,x_n)$ we want to test the null hypothesis $H_0: \mu = \mu_0$ vs $H_1: \mu \neq \mu_0$ (one-sample test).
We then base our bootstrap samples on $z_i = x_i - \bar{X}_{\text{sample}} + \mu_0$ and use the test statistic $t^{(b)} = t^*\left( Z^{(b)} \right) = \frac{\bar{Z}^{(b)} - \mu_0}{s_Z^{(b)} / \sqrt{n}}$
where $Z^{(b)} = (z_1^{(b)},\dots,z_n^{(b)})$ and $s_Z^{(b)} = \sqrt{\frac{1}{n-1} \sum_{i=1}^{n} \left( z_i^{(b)} - \bar{Z}^{(b)} \right)^2}$
and $t_{\text{obs}} = t^*\left( X_{\text{sample}} \right)$.

Here we assume that as the mean varies, the distributions are just translated versions of each other
(translation family) (the $H_0$ and $H_1$ distribution are just shifted versions of each other -
true for Gaussians but e.g. wrong for exponential distributions) where 
if our $x_i$ are lifetimes it might make sense to use
logged lifetimes as they are more likely to satisfy a translation or normal family
assumption \citep[chapter 16.4]{Efron1994}.

% make a figure with a and b part, one displaying bootstrap.svg, one paired_t_test.svg
\begin{figure}

    \centering
    \begin{subfigure}{0.8\textwidth}
      \centering
      \includesvg[width=.95\linewidth]{figures/bootstrap.svg}
      \caption{(Two-sided) Bootstrap Hypothesis Testing based on empirical distribution from bootstrap samples.}
      \label{fig:bootstrap}
    \end{subfigure}%



    \begin{subfigure}{0.8\textwidth}
      \centering
      \includesvg[width=.95\linewidth]{figures/paired_t_test.svg}
      \caption{(Two-sided) Standard Hypothesis test with distributional assumption.}
      \label{fig:distributional_assumptions}
    \end{subfigure}

    \caption{Bootstrap vs Distributional Assumption Hypothesis Testing.}
    \label{fig:tests}

\end{figure}

\subsubsubsection{Calculation of $p$ values in Bootstrap Hypothesis Testing}
Practically, we calculate estimates of the $p$ values (where for $p < \alpha$ we reject $H_0$) as
% cases for tight, left and two-sided test
\begin{equation}
    \begin{aligned}
        \hat{p}^*_{\text{right}} &= \frac{1}{B} \sum_{b=1}^{B} \mathbbm{1}_{\left[t^{(b)} \geq t_{\text{obs}} \right]} \\
        \hat{p}^*_{\text{left}} &= \frac{1}{B} \sum_{b=1}^{B} \mathbbm{1}_{\left[t^{(b)} \leq t_{\text{obs}} \right]} \\
        \hat{p}^*_{\text{two-sided}} &= \frac{1}{B} \left[ \min{\left\{ \sum_{b=1}^{B} \mathbbm{1}_{\left[t^{(b)} \geq t_{\text{obs}} \right]}, \sum_{b=1}^{B} \mathbbm{1}_{\left[t^{(b)} \leq t_{\text{obs}} \right]} \right\} } \right]
    \end{aligned}
\end{equation}

\subsubsubsection{Bootstrap Hypothesis Test on the Difference of Means}
Consider we have two samples $X^{(1)} = (x_1^{(1)},\dots,x_n^{(1)})$ and $X^{(2)} = (x_1^{(2)},\dots,x_n^{(2)})$. And we want to test the null-hypothesis

\begin{equation}
    H_0: \mu^{(2)} - \mu^{(1)} \leq c \quad \text{vs} \quad H_1: \mu^{(2)} - \mu^{(1)} > c
\end{equation}

where $c \in \mathbb{R}$. Assume the true means are $\mu^{(1)}$ and $\mu^{(2)}$. We can generate a
bootstrap distribution under $H_0$ by sampling from

\begin{equation}
    \begin{multlined}
        z^{(1)}_i = x^{(1)}_i - \bar{X}^{(1)}_{\text{sample}} - \frac{c}{2} \quad \text{and} \quad z^{(2)}_i = x^{(2)}_i - \bar{X}^{(2)}_{\text{sample}} + \frac{c}{2} \\   
        Z^{(1)} = (z_1^{(1)},\dots,z_n^{(1)}) \quad \text{and} \quad Z^{(2)} = (z_1^{(2)},\dots,z_n^{(2)})
    \end{multlined}
\end{equation}

and using the test statistic

\begin{equation}
    \begin{multlined}
    t_{\text{obs}} = \frac{\bar{X}^{(2)}_{\text{sample}} - \bar{X}^{(1)}_{\text{sample}} - c}{S_p\sqrt{\frac{1}{n^{(1)}} + \frac{1}{n^{(2)}}}} \\
    S_p = \sqrt{\frac{\left(n^{(1)} - 1\right)\left(S^{(1)}\right)^2 + \left(n^{(2)} - 1\right)\left(S^{(2)}\right)^2}{n^{(1)} + n^{(2)} - 2}} \\
    \end{multlined}
\end{equation}

with analogous calculations of $t^{(b)}$ on the bootstrap samples (the basic reasoning behind the denominator is that the scale on which means can be compared
is the standard deviation, without which a statement like the means are 1 apart is meaningless (very different statement for standard deviations
of the order of $10$ or $10^{-2}$)).

\yellowbox{\textbf{Implicit translational family assumption:} That the empirical distribution under $H_0$ would look like this, 
that it can be constructed based on shifting our sample data, assumes that the distribution of the sample is a translation family
(shifting the mean just shifts the distribution).}

\subsubsubsection{Bootstrap Hypothesis Test - Are two samples from different distributions?}
Consider we have samples
\begin{equation}
    \begin{array}{ll}
    X=\left\{x_1, \ldots, x_{N_X}\right\}, & X \sim F_X \\
    Y=\left\{y_1, \ldots, y_{N_Y}\right\}, & Y \sim F_Y
    \end{array}
\end{equation}
drawn from unknown distributions $F_X,F_Y$. Our aim is to test
\begin{equation}
    H_0: F_X = F_Y \quad \text { vs } \quad H_1: F_X \neq F_Y
\end{equation}
For testing if two samples possibly come from different distributions
(e.g. test results of girls and boys), the testing procedure is
\begin{enumerate}
    \item Formulate $H_0$, so $H_0: F_X = F_Y$.
    \item Specify the $\alpha$-level and test statistic $T$, which if $N_X = N_Y = N$ could be
    \begin{equation}
        T(X,Y) = \frac{\bar{\Delta}}{\sigma_\Delta \slash \sqrt{N}}, \quad \Delta_i = x_i - y_i
    \end{equation}
    \item \textbf{Bootstrapping:} From the union $Z = X \cup Y$ we draw $N_b$ samples of sizes $N_X$ and $N_Y$ respectively with replacement. For $b = 1,\dots,N_b$
    \begin{enumerate}
        \item Fraw samples with repition from $Z$
        \begin{equation}
            \begin{gathered}
                N_X \text{ draws with repetition from } Z \rightarrow \tilde{X}_b=\left\{\tilde{x}_1, \ldots, \tilde{x}_{N_X}\right\} \\
                N_Y \text{ draws with repetition from } Z \rightarrow \tilde{Y}_b=\left\{\tilde{y}_1, \ldots, \tilde{y}_{N_Y}\right\} \\
                \text{e.g. } \tilde{X}_b=\left\{x_1, y_2, x_1, y_5, \ldots\right\}, \tilde{Y}_b=\left\{y_1, y_4, x_3, y_8, \ldots\right\}
            \end{gathered}
        \end{equation}
        $H_0$ (\textit{what if they were from the same distribution}) is built-in as we draw from the union of the samples.
        \item Calculate the test statistic $t_b = T(\tilde{X}_b,\tilde{Y}_b)$.
    \end{enumerate}
    \item Step 3 yields a set of test statistics $t_1,\dots,t_{N_b}$ on which \textit{histogram} check if $t_{\text{obs}} = T(X,Y)$ falls 
    into the rejection region of $H_0$. We reject $H_0$ if
    \begin{equation}
        \begin{gathered}
            p_{E D F}\left(t \geq t_{\text {observed }}\right) \approx \frac{\# \text { of bootstrap } t \geq t_{\text {observed }}}{\text { bootstrap drawings } N_b}<\frac{\alpha}{2} \\
            \text { or } p_{E D F}\left(t \leq t_{\text {observed }}\right) \approx \frac{\# \text { of bootstrap } t \leq t_{\text {observed }}}{\text { bootstrap drawings } N_b}<\frac{\alpha}{2}
        \end{gathered}
    \end{equation}
    See figure \ref{fig:bootstrap} for an illustration.
\end{enumerate}

\subsection{Multiple testing problem | significance by chance}
\problem{Consider we are testing $100$ independent null 
hypotheses (e.g. benefits of alcohol on 100 health 
indicators) at $\alpha=0.05$. Then - only by chance - on average 
5 of them will lead to the wrong rejection of $H_0$ (type I error) (so to wrong 
significant results, e.g. alcohol is good for red body cell 
production). So we are likely to run into a type I error.}

\idea{We set a confidence level for the whole family of tests,
and scale down the $\alpha$-level for each test the more tests we do,
so we scale down the confidence that a single test is significant (/ will generalize
to independent data).}

We want to control the \textbf{confidence in the family of tests which is quantified by the 
family-wise error rate (FWER)}. The FWER is the probability of obtaining at least
one significant result just by chance (so when all null hypotheses are true).

\begin{equation}
    \begin{gathered}
        \hat{\alpha} = p(\# \{\text{accept } H_1 \mid H_0 \text{ is true}\} \geq 1) = 1 - (1-\alpha)^K \\
        \text{total number } K \text{ of hypotheses tested with significance level } \alpha \text{ per test}
    \end{gathered}
\end{equation}

\paragraph*{FWER control: Decrease $\alpha$ with $1 \slash K$} The so-called Bonferroni correction is
\begin{equation}
    \alpha_\text{per test} = \frac{\alpha_{\text{family}}}{K}, \quad \text{e.g. } \alpha_{\text{family}} = 0.05
\end{equation}

\note{This correction comes at the cost of increasing the probability for false negatives (type II error)
(so accepting the \textit{no-difference} statement when we really have a significant result), so decreasing
the statistical power of the test. Holm-Bonferroni is more powerful.
}

Alternatively one can also \textbf{control the false discovery rate} (the proportion of wrongly rejected null hypotheses (\textit{false-discoveries}) to all rejected hypothesis (\textit{discoveries})). See e.g.
Benjamin-Hochberg procedure.

\pagebreak