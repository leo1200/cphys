\part{Computational Statistics and Data Analysis}
\thispagestyle{plain}

Simulation is concerned 
with computing the evolution of a system
based on governing laws.

Statistics starts with data. It is concerned
with finding information in data, like distributions
of variables or relationships between them.

\bluebox{\textbf{Two Cultures of Statistical Modeling}: Consider
independent variables $\vec{x}$ and dependent variables $\vec{y}$.
How nature associates one with the other (e.g. CO2 with temperature)
is generally a black box. The two cultures of statistical modeling
\citep{breiman01}\footnote{Leo Breiman is probably best known for his work 
on Random Forests and Bagging. \enquote{A friend of mine, a prominent statistician from 
the Berkeley Statistics Department, visited me in Los Angeles in the late 1970s. 
After I described the decision tree method to him, his first question was, 
“What's the model for the data?”}} can be seen as
\begin{itemize}
    \item data-modeling culture: a simple interpretable stochastic model is assumed for the inside of the black box
    \item algorithmic modeling culture: more flexible models like decision trees or neural nets are used to predict $\vec{y}$ from $\vec{x}$
\end{itemize}
An addition to this dichotomy is what we already discussed
\begin{itemize}
    \item mechanistic modeling: derive a model from scientific theories
\end{itemize}
Of course those three cultures are extremes in model space, hybrids 
are possible.}